\documentclass[english]{reedthesis}

\usepackage[T1]{fontenc}
\usepackage{babel}

\usepackage{appendix}
\usepackage{amsfonts, amscd, amssymb, amsthm, amsmath}
\usepackage{mathtools} %xmapsto etc
\usepackage{pdfsync} %leaves makers for tex searching
\usepackage{enumitem}
\usepackage{booktabs}
\usepackage{complexity}
\usepackage{mleftright}

\usepackage[linesnumbered,algochapter]{algorithm2e}
\usepackage{biblatex}
\usepackage{imakeidx}
\usepackage{microtype}
\usepackage{tikz}
\usepackage{csquotes}

% FIXME: Really bad hack to get cleveref working
% This is unnecessary on literally every other LaTeX setup except mine
% When working again, will need to reset theorem defs to point to the thm counter
\usepackage{aliascnt}

\usepackage[colorlinks]{hyperref}
\usepackage[capitalize,noabbrev]{cleveref}

\usetikzlibrary{arrows.meta}

%%% Theorems %%%---------------------------------------------------------
\theoremstyle{plain}
\newtheorem{thm}{Theorem}[section]
\newaliascnt{lemma}{thm}
\newtheorem{lemma}[lemma]{Lemma}
\aliascntresetthe{lemma}
\newaliascnt{prop}{thm}
\newtheorem{prop}[prop]{Proposition}
\aliascntresetthe{prop}
\newaliascnt{cor}{thm}
\newtheorem{cor}[cor]{Corollary}
\aliascntresetthe{cor}
\theoremstyle{definition}
\newtheorem*{def*}{Definition}
\newaliascnt{defn}{thm}
\newtheorem{defn}[defn]{Definition}
\aliascntresetthe{defn}
\theoremstyle{remark}
\newtheorem{example}{Example}[thm]
\newtheorem{remark}[thm]{Remark}
\newtheorem{subrem}[example]{Remark}


\DeclareMathOperator{\ad}{ad}
\DeclareMathOperator{\Aut}{Aut}
\DeclareMathOperator{\ch}{ch}
\DeclareMathOperator{\col}{col}
\DeclareMathOperator{\diag}{diag}
\DeclareMathOperator{\dimn}{dim}
\DeclareMathOperator{\End}{End}
\DeclareMathOperator{\ev}{ev}
\def\f{\varphi}
\def\half{\hbox{$\frac12$}}
\DeclareMathOperator{\Hom}{Hom}
\DeclareMathOperator{\img}{img}
\DeclareMathOperator{\Inn}{Inn}
\DeclareMathOperator{\id}{id}
\DeclareMathOperator{\mdeg}{mdeg}
\DeclareMathOperator{\rad}{rad}
\DeclareMathOperator{\Rep}{Rep}
\DeclareMathOperator{\row}{row}
\DeclareMathOperator{\rk}{rank}
\def\normeq{\trianglelefteq}
\DeclareMathOperator{\nul}{nullity}
\DeclareMathOperator{\per}{per}
\DeclareMathOperator{\sgn}{sgn}
\DeclareMathOperator{\spn}{span}
\DeclareMathOperator{\supp}{supp}
\DeclareMathOperator{\Syl}{Syl}
\DeclareMathOperator{\Sym}{Sym}
\DeclareMathOperator{\tr}{tr}
\def\vep{\varepsilon}
\DeclareMathOperator{\lcm}{lcm}

\DeclarePairedDelimiter{\norm}{\lVert}{\rVert}
\DeclarePairedDelimiter{\abs}{\lvert}{\rvert}
\DeclarePairedDelimiter{\ang}{\langle}{\rangle}
\DeclarePairedDelimiter\ceil{\lceil}{\rceil}
\DeclarePairedDelimiter\floor{\lfloor}{\rfloor}

\DeclarePairedDelimiter\bra{\langle}{\rvert}
\DeclarePairedDelimiter\ket{\lvert}{\rangle}
\DeclarePairedDelimiterX\braket[2]{\langle}{\rangle}{#1\,\delimsize\vert\,\mathopen{}#2}
\DeclarePairedDelimiterX\braopket[3]{\langle}{\rangle}{#1\,\delimsize\vert\,\mathopen{}#2\,\delimsize\vert\,\mathopen{}#3}

\newcommand{\middlemid}{%
  \ensuremath{\;\middle\vert\;}
}

\newcommand{\dblang}[1]{%
  \ensuremath{\left\langle\!\left\langle#1\right\rangle\!\right\rangle}
}

\newcommand{\comment}[1]{%
  \text{\phantom{(#1)}} \tag{#1}%
}

\newcommand{\commath}[1]{%
  \phantom{(#1)} \tag{#1}%
}

\newcommand{\ipcp}[5]{%
  \ensuremath%
  \mleft[{\footnotesize\begin{array}{r l}
    \text{round complexity:} & #1 \\
    \text{PCP length:} & #2 \\
    \text{communication complexity:} & #3 \\
    \text{query complexity:} & #4 \\
    \text{soundness error:} & #5
  \end{array}}\mright]%
}

\newcommand{\ldipcp}[6]{%
  \ensuremath%
  \mleft[{\footnotesize\begin{array}{r l}
    \text{round complexity:} & #1 \\
    \text{PCP length:} & #2 \\
    \text{communication complexity:} & #3 \\
    \text{query complexity:} & #4 \\
    \text{oracle:} & #5 \\
    \text{soundness error:} & #6
  \end{array}}\mright]%
}

\newcommand{\mipstar}[4]{%
  \ensuremath%
  \mleft[{\footnotesize\begin{array}{r l}
    \text{number of provers:} & #1 \\
    \text{round complexity:} & #2 \\
    \text{communication complexity:} & #3 \\
    \text{soundness error:} & #4
  \end{array}}\mright]%
}

\newclass{\IPCP}{IPCP}
\makeatletter % complexity doesn't support hyphens by default...
\newcommand{\PZKPCP}{%
  \PZK\complexity@hyphenleft\PCP
}
\newcommand{\PZKIPCP}{%
  \PZK\complexity@hyphenleft\IPCP
}
\makeatother

\newlang{\OR}{OR}
\newlang{\OSAT}{O3SAT}
\newlang{\Prime}{Prime}

\addbibresource{bibliography.bib}

\makeindex[intoc]

\title{Thesis Draft: Algebrization}
\author{Patrick Norton}

\approvedforthe{Committee}
\thedivisionof{The Established Interdisciplinary Committee for \\}
\division{Mathematics and Computer Science}
\department{Mathematics and Computer Science}
\advisor{Zajj Daugherty}
\altadvisor{Adam Groce}

\begin{document}

\maketitle

\tableofcontents

% TODO: Abstract

\chapter*{Introduction}
\addcontentsline{toc}{chapter}{Introduction}

The $\P$ vs $\NP$ problem is perhaps the most important open problem in
complexity theory.

\chapter{Preliminaries}

% TODO: Do we need Hamming distance? I can't find them actually using it in CFGS22

\section{Turing machines}

% TODO: Do we even need to formally define a TM?

Central to our definitions of complexity is that of a Turing machine. This is
the most common mathematical model of a computer, and is the jumping-off point
for mant variants. There are many ways to think of a Turing machine, but the
most common is that of a small machine that can read and write to an
arbitrarily-long ``tape'' according to some finite set of rules. We give a more
formal definition below, and then we will attempt to take this definition into a
more manageable form.
\begin{defn}[{\cite[Def.\ 3.1]{Sip97}}]\label{def:TM}\index{Turing machine}
  A \emph{Turing machine} is a 7-tuple $(Q, \Sigma, \Gamma, \delta, q_{0}, q_{a}, q_{r})$ where
  $Q$, $\Sigma$, and $\Gamma$ are all finite sets and
  % TODO: Rephrase to separate out terminology and definitions?
  \begin{enumerate}
    \item $Q$ is the set of \emph{states},
    \item $\Sigma$ is the \emph{input alphabet},
    \item $\Gamma$ is the \emph{tape alphabet},
    \item $\delta: Q \times \Gamma \rightarrow Q \times \Gamma \times \{L, R\}$ is the \emph{transition function},
    \item $q_{0} \in Q$ is the \emph{start state},
    \item $q_{a} \in Q$ is the \emph{accept state},
    \item $q_{r} \in Q$ is the \emph{reject state}, with $q_{a} \ne q_{r}$.
  \end{enumerate}
\end{defn}

While we have this formalism here as a useful reference, even here we will most
frequently refer to Turing machines in a more intuitionisitc form. There are
several ways we will think about Turing machines.

The first way to think about a Turing machine is as a little computing box with
a tape. We let the box read and write to the tape, and each step it can move the
tape one space in either direction. At some point, the machine can decide it is
done, in which case we say it ``halts''; however it does not necessarily need to
halt. For this paper, we will only think about machines that \emph{do} halt, and
in particular we will care about how many it takes us to get there. Further, we
will use this informalism as a base from which we can define our Turing machine
variants intuitively, without needing to deal with the (potentially extremely
convoluted) formalism.

Another way we think about a Turing machine is as an algorithm. Perhaps the
foundational paper of modern computer science theory, the \emph{Church-Turing
  thesis}~\cite{Tur36}, states that any actually-computable algorithm has an
equivalent Turing machine, and vice versa. We will use this fact liberally; in
many cases we will simply describe an algorithm and not deal with putting it
into the context of a Turing machine. If we have explained the algorithm well
enough that a reader can execute it (as we endeavor to do), then we know a
Turing machine must exist.

\begin{defn}\label{def:nondeterministic-tm}\index{Turing machine!nondeterministic}
  A \emph{nondeterministic Turing machine} is % TODO
\end{defn}

\section{Complexity classes}

% TODO: Do I need to define big-O for this?
% TODO: Venn diagram of all the complexity classes in this section

Complexity classes are the main way we think about the hardness of problems in
computer science. A complexity class\index{complexity class} is a collection of
languages that all share a common level of difficulty.

We start with a relatively straightforward example of a complexity class: the
class of languages that a Turing machine can recognize. First, we need to
define what recognition is in order to make a complexity class out of it.

\begin{defn}[{\cite[Def.\ 3.2]{Sip97}}]\label{def:recognition}\index{recognize}
  A language $L$ is \emph{recognized} by a Turing machine $M$ if for all strings
  $s \in L$, $M$ halts in the accept state when given $s$ as input.
\end{defn}

Now, since our complexity classes are about \emph{languages}, we naturally wish
to extend our notion of recognition to a statistic on languages.

\begin{defn}\label{def:turing-recognizable}\index{Turing-recognizable language}
  A language $L$ is \emph{Turing-recognizable} (frequently just
  \emph{recognizable}) if it is recognized by some Turing machine.
\end{defn}

Now that we have a property of languages, it is straightforward for us to turn
it into a complexity class.

\begin{defn}\label{def:re}\index{RE@$\RE$}
  The class $\RE$ is the class of all Turing-recognizable languages.
\end{defn}
% TODO: Example (and non-example)

For most other classes, we want our Turing machines to halt on \emph{all}
inputs, not just those in the class. From a practical perspective, this is
useful because it tells us that we can be certain about whether any given string
is in the given language. From here on, we will generally care about how much of
some resource our machines take when making their decision, as opposed to
whether or not they can.

\subsection{Time complexity}\index{time complexity}

The most intuitive (and most important) notion of complexity is that of time
complexity. Time complexity is the answer of the question of how long it takes
to solve a problem. We begin with an abstract base for our time classes, and
will then introduce some specific ones that we care about.

\begin{defn}[{\cite[Def.\ 1.19]{AB09}}]\label{def:dtime}\index{DTIME@$\DTIME$}
  % FIXME: Reword to make more clear what $n$ is
  Let $f: \mathbb{N} \rightarrow \mathbb{N}$ be a function. The class $\DTIME(f(n))$ is the class of all
  problems computable by a deterministic Turing machine in $O(f(n))$ steps for
  some constant $c > 0$.
\end{defn}

While $\DTIME$ is a useful base to start from, it is rare that we deal with
$\DTIME$ classes directly. % TODO

\begin{defn}[{\cite[Def.\ 1.20]{AB09}}]\label{def:p}\index{P@$\P$}
  The complexity class $\P$ is the class
  \[
    \P = \bigcup_{c > 0}\DTIME(n^{c}).
  \]
\end{defn}

The class $\P$ is perhaps the most important complexity class. Mathematically,
we care about $\P$ because it is closed under composition: a polynomial-time
algorithm iterated a polynomial number of times is still in $\P$. Further, $\P$
turns out to generally be invariant under change of (deterministic) computation
model, which allows us to reason about $\P$ problems easily without needing to
resort to the formal definition of a Turing machine. More philosophically, $\P$
generally represents the set of ``efficient'' algorithms in the real
world.\footnote{It is worth mentioning that this is a \emph{mathematical}
  efficiency---there are plenty of algorithms in $\P$ that a real-world computer
  scientist would never dare to call efficient.}

\begin{example}\label{ex:polynomial-is-p}
  The language
  \[
    \{(p, x, y) \mid p \text{ a polynomial and } p(x) = y\}
  \]
  is in $\P$. We can calculate whether a string is in this language by
  calculating $p(x)$ (which we can do efficiently), and then comparing it to
  $y$.
\end{example}

As we have defined $\P$ in terms of $\DTIME$, the question arises of whether
there is an equivalent in terms of $\NTIME$. Naturally, there is, and we call it
$\NP$.

\begin{defn}[{\cite[Cor.\ 7.22]{Sip97}}]\label{def:np}\index{NP@$\NP$}
  The complexity class $\NP$ is the class
  \[
    \NP = \bigcup_{c > 0}\NTIME(n^{c}).
  \]
\end{defn}

While this definition demonstrates how $\NP$ is similar to $\P$, there are other
equivalent ones that we can use. In particular, we very often like to think of
$\NP$ in terms of deterministic \emph{verifiers}. Since nondeterministic
machines do not exist in real life, this definition gives a practical meaning to
$\NP$.

\begin{example}\label{ex:sat-is-np}
  The language $\SAT$ is the language of Boolean formulas with at least one
  solution. $\SAT$ is in $\NP$: we can nondeterministically pick a potential
  solution and then evaluate our formula (which can be done efficiently); there
  will be an accepting path if and only if a solution to the formula exists.
\end{example}

\begin{thm}[{\cite[Def.\ 7.19]{Sip97}}]\label{thm:np-verifier}
  $\NP$ is exactly the class of all languages verifiable by a $\P$-time Turing
  machine.
\end{thm}

% TODO: Should I prove this?

\begin{example}\label{ex:sat-np-verifier}
  The language $\SAT$ we defined in \cref{ex:sat-is-np} can be verified
  efficiently, where the certificate is an accepting set of variables. Since we
  can evaluate a Boolean formula efficiently, if we already have an accepting
  set of variables we can therefore verify it in $\P$.
\end{example}

The next step up from polynomial complexities is that of exponential
complexities. For these, instead of having the classes bounded above by a
polynomial, we have the classes bounded above by $2$ to the power of a
polynomial. While we use $2$ as the base, the value of the base turns out not to
matter since for any $a, b > 1$,
\begin{equation}
  a^{n^{c}} = b^{n^{c}\log_{b}(a)} \in O\mleft(b^{n^{c+1}}\mright).
\end{equation}

\begin{defn}[{\cite[\defaultS 2.6.2]{AB09}}]\label{def:exp}\index{EXP@$\EXP$}
  The complexity class $\EXP$ is the class
  \[
    \EXP = \bigcup_{c > 0}\DTIME\mleft(2^{n^{k}}\mright).
  \]
\end{defn}

\begin{defn}[{\cite[\defaultS 2.6.2]{AB09}}]\label{def:nexp}\index{NEXP@$\NEXP$}
  The complexity class $\NEXP$ is the class
  \[
    \NEXP = \bigcup_{c > 0}\NTIME\mleft(2^{n^{k}}\mright).
  \]
\end{defn}

It is immediate that $\P \subseteq \EXP$ and $\NP \subseteq \NEXP$ (since the exponential
classes allow the use of more of the same resource). Of slightly less-trivial
interest is the relationship between $\NP$ and $\NEXP$.

\begin{thm}\label{thm:np-exp}
  $\NP \subseteq \EXP$.
\end{thm}

\begin{proof}
  If a nondeterministic machine solves a problem in $p(n)$ steps, it follows
  that the total number of branches is less than $a^{p(n)}$, where $a$ is the
  maximum number of branches for a node within the machine. Hence, we can
  simulate the machine deterministically by simply enumerating every branch,
  giving us a total computation time of $p(n)a^{p(n)}$, which is in
  $O(2^{q(n)})$ for some other polynomial $q(n)$. Hence any $\NP$ problem is in
  $\EXP$.
\end{proof}

It is perhaps illustrative to see an example of a problem in $\NEXP$. % TODO

% https://www.cs.umd.edu/~jkatz/complexity/f11/lecture26-dapon.pdf
\begin{defn}[{\cite[Def.\ 14.1]{CFGS22}}]\label{def:oracle-3sat}\index{O3SAT@$\OSAT$}
  The \emph{oracle 3-satisfiability problem}, denoted $\OSAT$, is the language
  of all triplets $(r, s, B)$, where $r, s \in \mathbb{N}^{+}$ and
  $B: \{0, 1\}^{r+3s+3} \rightarrow \{0, 1\}$ a boolean function, such that there exists a
  boolean function $A: \{0, 1\}^{s} \rightarrow \{0, 1\}$ having the property that for all
  $z \in \{0, 1\}^{r}$ and $b_{1}, b_{2}, b_{3} \in \{0, 1\}^{s}$,
  \begin{equation*}
    B(z, b_{1}, b_{2}, b_{3}, A(b_{1}), A(b_{2}), A(b_{3})) = 1.
  \end{equation*}
\end{defn}

% TODO: Cite
\begin{thm}\label{thm:o3sat-in-nexp}
  $\OSAT \in \NEXP$.
\end{thm}

\begin{proof}
  We present the following non-deterministic algorithm to determine if
  $(r, s, B) \in \OSAT$:
  \begin{algorithm}[H]
    \KwIn{A triplet $(r, s, B)$}
    \KwOut{Whether or not $(r, s, B) \in \OSAT$}
    Nondeterministically choose a function $A: \{0, 1\}^{s} \rightarrow \{0, 1\}$\;
    \For{$z \in \{0, 1\}^{r}$}{
      \For{$b_{1}, b_{2}, b_{3} \in \{0, 1\}^{s}$}{
        Compute $B(z, b_{1}, b_{2}, b_{3}, A(b_{1}), A(b_{2}), A(b_{3}))$\;
        \If{the above is not $1$}{
          \KwRet{0}\;
        }
      }
    }
    \KwRet{1}\;
    \caption{A $\NEXP$-time algorithm for determining $\OSAT$}\label{alg:osat-nexp}
  \end{algorithm}
  % TODO: Explain why this is NEXP
  % FIXME: There's something *extremely* sus here: I think this is in AP since
  % we can nondeterministically choose z, b1, b2, b3 and then the only
  % substantial computation is of B--why does that not mean O3SAT \in EXP and hence
  % EXP = NEXP?
  % Wait... even more sus: AP = PSPACE so that implies PSPACE = NEXP
\end{proof}

\subsection{Space complexity}\index{space complexity}

In addition to time complexity, the an additional notion of complexity is that
of space complexity. Space complexity is the question of how much space on its
memory tape a machine needs in order to compute a problem. In many ways, our
definitions of space complexity are analagous to those for time complexity that
we have already defined. In particular, $\DSPACE$ will correspond nicely to
$\DTIME$, and $\NSPACE$ to $\NTIME$.

\begin{defn}[{\cite[Def.\ 4.1]{AB09}}]\label{def:dspace}\index{DSPACE@$\DSPACE$}
  Let $f: \mathbb{N} \rightarrow \mathbb{N}$ be a function. A language $L$ is in $\DSPACE(f(n))$ if there
  exists a deterministic Turing machine $M$ such that the number of locations on
  the tape that are non-blank at some point during the execution of $M$ is
  in $O(f(n))$.
\end{defn}

In the same way as we have defined $\DSPACE$ for deterministic machines, we now
need to define $\NSPACE$ for nondeterministic machines.

\begin{defn}[{\cite[Def. 4.1]{AB09}}]\label{def:nspace}\index{NSPACE@$\NSPACE$}
  Let $f: \mathbb{N} \rightarrow \mathbb{N}$ be a function. A language $L$ is in $\NSPACE(f(n))$ if there
  exists a nondeterministic Turing machine $M$ such that the number of locations
  on the tape that are non-blank at some point during the execution of $M$ is in
  $O(f(n))$.
\end{defn}

Analagously to $\P$ and $\NP$, our two main classes of space complexity are
$\PSPACE$ and $\NPSPACE$.

\begin{defn}[{\cite[Def.\ 4.5]{AB09}}]\label{def:pspace}\index{PSPACE@$\PSPACE$}
  The complexity class $\PSPACE$ is the class
  \[
    \PSPACE = \bigcup_{c > 0}\DSPACE(n^{c}).
  \]
\end{defn}

\begin{defn}[{\cite[Def.\ 4.5]{AB09}}]\label{def:npspace}\index{NPSPACE@$\NPSPACE$}
  The complexity class $\NPSPACE$ is the class
  \[
    \NPSPACE = \bigcup_{c > 0}\NSPACE(n^{c}).
  \]
\end{defn}

Unlike with $\P$ and $\NP$, the relationship between $\PSPACE$ and $\NPSPACE$ is
well known. Due to the complexity of the proof of the theorem, we will not prove
it here, as it is mostly not relevant to what we will be doing.

\begin{thm}[{Savitch's theorem;~\cite{Sav70}}]\label{thm:savitch}\index{Savitch's theorem}
  $\PSPACE = \NPSPACE$.
\end{thm}

Upon seeing this, one might ask why it is that we believe $\P \ne \NP$ if we know
that $\PSPACE = \NPSPACE$, given they are defined analogously. The answer to
this question boils down to the fact that we are able to reuse space, while we
are not able to reuse time. Space on the tape that is no longer needed can be
overwritten, while time that is no longer needed is gone forever.

Since $\PSPACE$ and $\NPSPACE$ are equal classes, it is relatively rare to see
$\NPSPACE$ referred to. Here, we will only refer to it when it makes a class
relationship clearer; most frequently when comparing $\NPSPACE$ to some other
nondeterministic class.

\begin{example}\label{ex:regex-is-pspace}
  The language
  \[
    \{(x, y) \mid x, y \text{ regexes that accept the same set of strings}\}
  \]
  is in $\PSPACE$. % TODO: Explain why
\end{example}

% TODO: If/when we pull random/quantum stuff in here, probably worth defining
% BPP and BQP (QMA might need its own section?)

% TODO: Draw a graph of all these complexity inclusions

\subsection{Completeness}

% TODO: Not a huge fan of this paragraph
Even within a complexity class, not all problems are created equal. The notion
of \emph{completeness} gives us a mathematically-rigorous way to talk about
which problems in a class are the hardest. Since putting upper bounds on hard
problems naturally puts those same bounds on any easier problems, complete
problems can be useful in reasoning about the relationship between complexity
classes.

\begin{defn}[{\cite[Def.\ 7.29]{Sip97}}]\label{def:p-reduction}\index{polynomial-time reduction}
  A language $A$ is \emph{polynomial-time reducible} to a language $B$ if a
  polynomial-time computable function $f: \Sigma^{*} \rightarrow \Sigma^{*}$ exists such that for
  all $w \in \Sigma^{*}$, $w \in A$ if and only if $f(w) \in B$.
\end{defn}

Polynomial-time reductions are important because they give us a way to say that
$A$ is \emph{no harder} than $B$. In particular, if we have an algorithm $M$
that determines $B$, we can construct the following algorithm that determines
$A$ with only a polynomial amount of additional work:

\begin{algorithm}[H]
  \KwIn{A string $w \in \Sigma^{*}$}
  \KwOut{Whether $w \in A$}
  Compute $f(w)$\;
  Use $M$ to check whether $f(w) \in B$\;
  \KwRet{the result of $M$}\;
  \caption{An algorithm to reduce $A$ to $B$}
\end{algorithm}

\begin{defn}[{\cite[Def.\ 7.34]{Sip97}}]\label{def:np-complete}\index{NP-complete@$\NP$-complete}
  A language $L$ is $\NP$-complete if $L \in \NP$ and every $A \in \NP$ is
  polynomial-time reducible to $L$.
\end{defn}

This is a practical use of our polynomial-time reductions: since an
$\NP$-complete language has a reduction from every other language in $\NP$, it
follows that it is \emph{at least as hard} as any other language in $\NP$. Of
particular interest to complexity theorists is the fact that $\P = \NP$ if and
only if \emph{any} $\NP$-complete language is in $\P$.

\begin{example}\label{ex:sat-is-complete}\index{Cook-Levin theorem}
  A famous result of Cook~\cite{Cook71}, also proved around the same time by
  Levin and thus called the \emph{Cook-Levin theorem}, is that the $\SAT$
  problem we defined earlier in \cref{ex:sat-is-np} is $\NP$-complete.
\end{example}

% TODO: Talk about why we use completeness (if it's true for anything here, it's
% true for everything in NP)
The notion of completeness is very important to complexity theorists. Since
these are the ``hardest'' problems in $\NP$, this means that if we can do
anything interesting to an $\NP$-complete problem, we can leverage these
reductions to do that interesting thing to \emph{any} other problem in $\NP$
with only a little (i.e.\ polynomial) more effort. This will come in especially
handy when we want to prove that complexity classes are equal or that $\NP$ is a
subset of some other complexity class--since most complexity classes allow for
things to change polynomially, we only need to prove that a single
$\NP$-complete element is in the other class for the subset relation to follow.

% TODO: Cite NEXP not being closed under EXP reductions
Along with completeness for $\NP$, we have a notion of completeness for $\NEXP$.
While you might expect that the reducibility constraints might loosen (i.e.\
allow more complex reductions) since $\NEXP$ is more complex for $\NP$, but this
turns out not to be the case. In particular, while it might initially seem
logical to allow for $\EXP$-reductions, it turns out that $\NEXP$ is not closed
under $\EXP$-reductions, which makes a notion of completeness challenging.
Despite this, we can still learn interesting things about $\NEXP$ by studing
completeness under polynomial reductions.

\begin{defn}\label{def:nexp-complete}\index{NEXP-complete@$\NEXP$-complete}
  A language $L$ is $\NEXP$-complete if $L \in \NEXP$ and every $A \in \NEXP$ is
  polynomial-time reducible to $\NEXP$.
\end{defn}

$\NEXP$-completeness has many of the same nice properties of $\NP$-completeness.
Of particular interest to us will again be the ease with which
$\NEXP$-completeness allows us to determine subset relations, simply by proving
the inclusion of a single complete language.

\begin{thm}[{\cite[Proposition 4.2]{BFL90}}]\label{thm:o3sat-nexp-complete}
  The language $\OSAT$ (as defined in \cref{def:oracle-3sat}) is
  $\NEXP$-complete.
\end{thm}

\begin{proof}
  We demonstrated in \cref{thm:o3sat-in-nexp} that $\OSAT \in \NEXP$, so all that
  remains is to prove that reductions exist for every $\NEXP$ language. Let
  $L \in \NEXP$, and let $x \in \{0, 1\}^{n}$. We aim to construct an algorithm in
  $\P^{\OSAT}$ that computes $L$.
  % TODO
\end{proof}

Just as we have $\NP$-completeness and $\NEXP$-completeness for time complexity,
we also have notions of completeness for space complexity. Since
$\PSPACE = \NPSPACE$, instead of calling the class $\NPSPACE$-complete, we call
it $\PSPACE$-complete.

\begin{defn}[{\cite[Def.\ 8.8]{Sip97}}]\label{def:pspace-complete}\index{PSPACE-complete@$\PSPACE$-complete}
  A language $L$ is $\PSPACE$-complete if $L \in \PSPACE$ and every $A \in \PSPACE$
  is polynomial-time reducible to $\NP$.
\end{defn}

While this definition is mostly analagous to that of $\NP$-completeness, one
might wonder why we use a time complexity for our reduction when $\PSPACE$ is a
space-complexity class. This is because if we were to use space complexity, we
would want to use $\PSPACE$-reductions, but that would make every language in
$\PSPACE$ trivially $\PSPACE$-complete. Since that is not a useful definition,
we instead restrict ourselves to polynomial-time reductions.

\begin{example}
  A result of Stockmeyer and Meyer~\cite{SM73} is that the language we defined
  in \cref{ex:regex-is-pspace} is $\PSPACE$-complete.
\end{example}

\subsection{Counting complexity}\index{counting complexity}

\begin{defn}\label{def:counting-problem}\index{counting problem}
  A \emph{counting problem} is % TODO
\end{defn}

\begin{defn}[{\cite[Def.\ 9.2]{AB09}}]\label{def:hash-p}\index{P#@$\#\P$}
  The class $\#\P$ is the class of functions $f: \{0, 1\}^{*} \rightarrow \mathbb{N}$ such that
  there exists a polynomial $p: \mathbb{N} \rightarrow \mathbb{N}$ and a polynomial-time Turing machine $M$
  such that for every $x \in \{0, 1\}^{*}$,
  \begin{equation}
    f(x) = \abs*{\mleft\{y \in \{0, 1\}^{p(\abs{x})} \middlemid M(x, y) = 1\mright\}}.
  \end{equation}
\end{defn}

\begin{defn}\label{def:hash-p-complete}\index{P#-complete@$\#\P$-complete}
  A language is $\#\P$-complete if % TODO: Do I want to use the definition with FP?
\end{defn}

\begin{defn}\label{def:hash-sat}\index{SAT#@$\#\SAT$}
  The function $\#\SAT$ is the function that, given a Boolean formula $\phi$,
  returns the number of distinct assignments such that $\phi$ is true.
\end{defn}

\begin{thm}\label{thm-hash-sat-is-complete}
  $\#\SAT$ is $\#\P$-complete.
\end{thm}

\section{Polynomials}\label{sec:polynomial}

\begin{defn}[{\cite{Knu92}}]\label{def:iverson-bracket}\index{Iverson bracket}
  Let $P$ be a mathematical statement. The function $[P]$ is the the function
  \begin{equation}\label{eqn:iverson-bracket}
    [P] = \begin{cases}
      1 & P \text{ is true} \\
      0 & \text{otherwise.}
    \end{cases}
  \end{equation}
  This is called the \emph{Iverson bracket}, after its inventor Kenneth Iverson,
  who originally included it in the programming language APL\footnote{The
    original notation used parentheses, but square brackets are much less
    ambiguous, so that has become the standard and what we will use
    here.}~\cite[11]{APL}.
\end{defn}

\begin{example}
  Using the Iverson bracket, the Kronecker delta function can be defined as
  \[
    \delta_{ij} = [i = j].
  \]
\end{example}
% TODO: Transition sentence

Much of our work will deal with multivariate polynomials. For a given field
$\mathbb{F}$, we will denote the set of $m$-variable polynomials over
$\mathbb{F}$ with $\mathbb{F}[x_{1, \ldots, m}]$.

\begin{defn}[{\cite[8]{AW09}}]\label{def:mdeg}\index{multidegree}
  The \emph{multidegree} of a multivariate polynomial $p$, written $\mdeg(d)$,
  is the maximum degree of any variable $x_{i}$ of $p$.
\end{defn}

It is worth noting that for monovariate polynomials, multidegree and degree
coincide. The difference between multidegree and degree is subtle, but
important. We shall illustrate the difference with a simple example.

\begin{example}
  Consider the polynomial $x_{1}^{2}x_{2} + x_{2}^{2}$. The multidegree of this
  polynomial is 2, while its degree is 3.
\end{example}

We denote by $\mathbb{F}[x_{1, \ldots, m}^{\le d}]$ the subset of
$\mathbb{F}[x_{1, \ldots, m}]$ of polynomials with multidegree at most $d$. We also
need two special cases of these polynomials, which we will want to quickly be
able to reference throughout the paper.

\begin{defn}[{\cite[8]{AW09}}]\label{def:mlin}\index{multilinear}\index{multiquadratic}
  A polynomial is \emph{multilinear} if it has multidegree at most 1. Similarly,
  a polynomial is \emph{multiquadratic} if it has multidegree at most 2.
\end{defn}

From here, we need to define the notion of an \emph{extension polynomial}. This
gives the ability to take an arbitrary multivariate function defined on a subset
of a field and extend it to be a multivariate polynomial over the \emph{whole}
field.

\begin{defn}[{\cite[8]{AW09}}]\label{def:ext-poly}\index{extension polynomial}
  Let $\mathbb{F}$ be a finite field, $H \subseteq \mathbb{F}$, $m \in \mathbb{N}$ a number, and
  $f: H^{m} \rightarrow \mathbb{F}$ be a function. An \emph{extension polynomial} of $f$
  is any polynomial $f' \in \mathbb{F}[x_{1, \ldots, m}]$ such that $f(h) = f'(h)$ for
  all $h \in H$.
\end{defn}
% TODO: Examples

It turns out that this polynomial needs only to be of a surprisingly low
multidegree. Since polynomials of lower degree are generally easier to compute,
we would like to have some measure of what a ``small'' polynomial actually is in
this context.

% TODO: Rewrite \hat{f} as \tilde{f} to agree with AW09
\begin{defn}[{\cite[\defaultS 5.1]{CFGS22}}]\label{def:low-deg-ext}\index{low-degree extension}
  Let $\mathbb{F}$ be a finite field, $H \subseteq \mathbb{F}$, $m \in \mathbb{N}$ a number, and
  $f: H^{m} \rightarrow \mathbb{F}$ be a function. A \emph{low-degree extension} $\hat{f}$
  of $f$ is an extension of $f$ with multidegree at most $\abs{H} - 1$.
\end{defn}

% TODO: Cite these statements
It turns out that this is the minimum possible degree of any extension
polynomial. Further, it turns out that for any $f$, there is a \emph{unique}
low-degree extension. Neither of these statements are particularly important for
our further work, so we will not endeavor to prove them here. Something of
practical use to us is an explicit formula for the low-degree extension, which
we shall now calculate.

\begin{thm}[{\cite[\defaultS 5.1]{CFGS22}}]\label{thm:low-deg-ext-exists}
  Let $\mathbb{F}$ be a finite field, $H \subseteq \mathbb{F}$, $m \in \mathbb{N}$ a number, and
  $f: H^{m} \rightarrow \mathbb{F}$. Then a low-degree extension $\hat{f}$ of $f$ is the
  function
  \begin{equation}
    \hat{f}(x) = \sum_{\beta \in H^{m}}\delta_{\beta}(x)f(\beta),
  \end{equation}
  where $\delta$ is the polynomial
  % TODO? Flip x and y here
  \begin{equation}\label{eqn:delta-poly}
    \delta_{x}(y) = \prod_{i = 1}^{m}\mleft(\sum_{\omega \in H}\mleft(
        \prod_{\gamma \in H \setminus \{\omega\}}\frac{(x_{i} - \gamma)(y_{i} - \gamma)}{(\omega - \gamma)^{2}}
      \mright)\mright).
  \end{equation}
\end{thm}

\begin{proof}
  First, we must show $\hat{f}$ has multidegree $\abs{H} - 1$. First, note that
  $\hat{f}$ is a linear combination of some $\delta_{x}$es; hence asking about the
  multidegree of $\hat{f}$ is really just asking about the multidegree of
  $\delta_{x}$. Looking at $\delta_{x}$, the innermost product has $\abs{H} - 1$ terms,
  each with the same $y_{i}$; thus those terms have multidegree $\abs{H} - 1$.
  Summing terms preserves their multidegree, and the outer product iterates over
  the variables, thus it preserves multidegree as well. Thus, $\delta_{x}$ has
  multidegree $\abs{H} - 1$.

  % TODO: Turn this into a lemma?
  To understand why $\hat{f}(x)$ agrees with $f(x)$ on $H$, we first should look
  at $\delta_{\beta}(x)$. In particular, for all $x, y \in H^{m}$,
  \begin{equation}\label{eqn:delta-is-delta}
    \delta_{y}(x) = [x = y] = \delta_{xy}.
  \end{equation}
  This can be shown through some algebra which we have worked through in full
  detail in \cref{app:ext-poly}. This is the reason why we have named the
  polynomial in \cref{eqn:delta-poly} as we have; it functions as the Kronecker
  delta function over the set $H^{m}$.

  Taking the above statement, we get that for all $x \in H^{m}$, the only nonzero
  term of $\hat{f}(x)$ is the term where $\beta = x$; thus $\hat{f}(x) = f(x)$.
  Hence, $\hat{f}$ is a low-degree extension of $f$.
  % TODO: Does this even need a proof or can we leave it as a claim?
\end{proof}

Of particular interest to us will be the case of low-degree extensions where
$H = \{0, 1\}$. Since every field contains both $0$ and $1$, this will allow us
to construct a set consisting of an extension for \emph{every} field. Further,
since $\abs{H} = 2$ here, it means our low-degree extensions will be
multilinear. Not only do we thus constrain our polynomial to have a very low
multidegree, the $\delta$ function also dramatically simplifies in this case, which
makes it much easier to reason about.

\begin{cor}[{\cite[\defaultS 4.1]{AW09}}]\label{cor:low-degree-boolean}
  Let $\mathbb{F}$ be a finite field, $m \in \mathbb{N}$ a number, and
  $f: \{0, 1\}^{m} \rightarrow \mathbb{F}$. Then
  \begin{equation}\label{eqn:low-deg-ext-small}
    \hat{f}(x) = \sum_{\beta \in \{0, 1\}^{m}}\delta_{\beta}(x)f(\beta)
  \end{equation}
  is a low-degree extension of $f$, where $\delta$ is the polynomial
  \begin{equation}\label{eqn:delta-poly-small}
    \delta_{y}(x) = \mleft(\prod_{i:y_{i}=1}x_{i}\mright)\mleft(\prod_{i:y_{i}=0}(1 - x_{i})\mright).
  \end{equation}
\end{cor}
Note that in the product bound $i:y_{i} = 1$, we mean the product over all
numbers $i$ such that $y_{i} = 1$.
% TODO: Prove?

As we can see, the form of $\delta$ in \cref{eqn:delta-poly-small} is much more
manageable than the form in \cref{eqn:delta-poly}, and it is perhaps more
immediately apparent here why $\delta$ has the property it does. Further, since this
equation has no division, it turns out that it is valid for arbitrary
(non-trivial) rings, while the more complex equation is only valid for fields.
We show the algebra that brings us from the first to the second in
\cref{app:ext-poly}.

The form of $\delta_{y}$ defined in \cref{eqn:delta-poly-small} has further use to us
than just being simpler. In particular, these $\delta_{y}$ form a basis of
multilinear polynomials (and hence a generating set for the ring of all
polynomials). This is a particularly useful basis because it allows us to reason
about multilinear polynomials based solely on their outcomes on the Boolean
cube.\footnote{As an aside, this fact provides a relatively slick proof of the
  special case of our unproven statement earlier that low-degree extensions are
  both of minimal degree and unique.}

% TODO: Find better statement of this theorem
\begin{thm}[{\cite[\defaultS 4.1]{AW09}}]
  For any field $\mathbb{F}$, the set $\{\delta_{x} \mid x \in \{0, 1\}^{n}\}$ forms a
  basis for the vector space of multilinear polynomials
  $\mathbb{F}^{n} \rightarrow \mathbb{F}$.
\end{thm}

\begin{proof}
  Since $\delta_{y}(x) = 0$ for all $y \ne x \in \{0, 1\}^{n}$, it follows that the only
  way to get
  \begin{equation}
    \sum_{y \in \{0, 1\}^{n}}a_{y}\delta_{y} = 0
  \end{equation}
  is to have each $a_{y} = 0$. Hence the set of $\delta_{x}$ is linearly independent.
  Further, the vector space of multilinear polynomials has $2^{n}$ dimensions;
  since there are $2^{n}$ distinct $\delta_{x}$ polynomials, it follows that they
  form a basis.
\end{proof}

Now, we can use this fact to prove some cases where our low-degree extensions
turn out to have a particularly low degree. Unfortunately, these do have a lot
of qualifiers to them, but they will be useful in later theorems (in particular
\cref{lem:multiquad-adversary}).

\begin{thm}[{\cite[Theorem 4.3]{AW09}}]\label{thm:multiquad-extension}
  Let $\mathbb{F}$ be a field and $Y \subseteq \mathbb{F}^{n}$ be a set of $t$ points
  $y_{1}, \ldots, y_{t}$. Then for at least $2^{n} - t$ Boolean points
  $w \in \{0, 1\}^{n}$, there exists a multiquadratic extension polynomial
  $p: \mathbb{F}^{n} \rightarrow \mathbb{F}$ such that
  \begin{enumerate}
    \item $p(y_{i}) = 0$ for all $i \in [t]$,
    \item $p(w) = 1$,
    \item $p(z) = 0$ for all Boolean $z \ne w$.
  \end{enumerate}
\end{thm}

\begin{proof}
  % TODO
  % NOTE: AW09 Lemma 4.2 is *almost* a corollary of our Theorem 1.3.5, except
  % for the fact that not all our points are in \{0, 1\}^n
\end{proof}

% TODO: Do we want this here?
% TODO: Add preceding lemmas
% TODO: Come up with descriptive names for these things (adversary polynomials?)
\begin{lemma}[{\cite[Lemma 4.5]{AW09}}]\label{lem:multiquad-adversary}
  Let $\mathcal{F}$ be a collection of fields. Let $f: \{0, 1\}^{n} \rightarrow \{0, 1\}$ be a
  Boolean function, and for every $\mathbb{F} \in \mathcal{F}$, let
  $p_{\mathbb{F}}: \mathbb{F}^{n} \rightarrow \mathbb{F}$ be a multiquadratic polynomial
  over $\mathbb{F}$ extending $f$. Also let $\mathcal{Y}_{\mathbb{F}} \in \mathbb{F}^{n}$
  for each $\mathbb{F} \in \mathcal{F}$, and define
  $t = \sum_{\mathbb{F} \in \mathcal{F}}\abs{\mathcal{Y}_{\mathbb{F}}}$.

  Then, there exists a subset $B \subseteq \{0, 1\}^{n}$, with $\abs{B} \le t$, such that
  for all Boolean functions $f': \{0, 1\}^{n} \rightarrow \{0, 1\}$ that agree with $f$ on
  $B$, there exist multiquadratic polynomials
  $p_{\mathbb{F}}':\mathbb{F}_{n} \rightarrow \mathbb{F}$ (one for each $\mathbb{F} \in \mathcal{F}$)
  such that
  \begin{enumerate}
    \item $p_{\mathbb{F}}'$ extends $f'$, and
    \item $p_{\mathbb{F}}'(y) = p_{\mathbb{F}}(y)$ for all $y \in \mathcal{Y}_{\mathbb{F}}$.
  \end{enumerate}
\end{lemma}

\begin{proof}
  Call $z \in \{0, 1\}^{n}$ \emph{good} if for every $\mathbb{F} \in \mathcal{F}$ there exists
  a multiquadratic poylnomial $u_{\mathbb{F},z}: \mathbb{F}^{n} \rightarrow \mathbb{F}$
  such that
  \begin{enumerate}[label=\alph*.]
    \item\label{item:zero-in-y} $u_{\mathbb{F},z}(y) = 0$ for all
          $y \in \mathcal{Y}_{\mathbb{F}}$,
    \item\label{item:delta-one} $u_{\mathbb{F},z}(z) = 1$, and
    \item\label{item:delta-zero} $u_{\mathbb{F},z} = 0$ for all
          $w \in \{0, 1\}^{n} \setminus \{z\}$.
  \end{enumerate}
  % TODO: Rephrase w/more explanation
  From \cref{thm:multiquad-extension}, each $\mathbb{F} \in \mathcal{F}$ can prevent at most
  $\abs{\mathcal{Y}_{\mathbb{F}}}$ points from being good. Since
  $t = \abs{\mathcal{Y}_{\mathbb{F}}}$, there are at least $2^{n} - t$ good points.

  Let $G$ be the set of good points, and thus let $B = \{0, 1\}^{n} \setminus G$ be the
  set of not-good points. Define
  \begin{equation}\label{eqn:p-prime}
    p'_{\mathbb{F}}(x) = p_{\mathbb{F}}(x) + \sum_{z \in G}(f'(z) - f(z))u_{\mathbb{F},z}(x).
  \end{equation}
  Now, all we need is to show that $p'_{\mathbb{F}}(x)$ satisfies the two
  conditions from the theorem statement.

  First, we show that $p_{\mathbb{F}}'$ extends $f'$; that is,
  $p_{\mathbb{F}}'(x) = f'(x)$ for all $x \in \{0, 1\}^{n}$. There are two cases
  here: $x \in G$ and $x \in B$. If $x \in B$, then the sum term of \cref{eqn:p-prime}
  is $0$; hence $p'_{\mathbb{F}}(x) = p_{\mathbb{F}}(x)$. Since
  $p_{\mathbb{F}}(x)$ extends $f(x)$, and since $f(x) = f'(x)$ on $B$, this
  means $p'_{\mathbb{F}}(x) = f'(x)$. If $x \in G$, then the only term of the sum
  where $u_{\mathbb{F},z}(x)$ is nonzero is where $x = G$, as per
  \cref{item:delta-one,item:delta-zero} above. Hence, we have
  \[
    p'_{\mathbb{F}}(x) = p_{\mathbb{F}}(x) + f'(x) - f(x),
  \]
  and since $p_{\mathbb{F}}(x) = f(x)$, it follows that
  $p'_{\mathbb{F}}(x) = f'(x)$.

  Next, we show that $p_{\mathbb{F}}'(y)$ and $p_{\mathbb{F}}(y)$ agree for all
  $y \in \mathcal{Y}_{\mathbb{F}}$. Since by \cref{item:zero-in-y} above, we have that
  $u_{\mathbb{F},z}(y) = 0$, it follows that the entire sum term is zero.
  Therefore, $p'_{\mathbb{F}}(y) = p_{\mathbb{F}}(y)$ for all
  $y \in \mathcal{Y}_{\mathbb{F}}$.

  As such, we have constructed a polynomial $p'_{\mathbb{F}}$ and a set $B$ that
  satisfy our conditions of the theorem.
\end{proof}

% TODO: Unpack all that

% NOTE: JKRS09 is actually even stronger than this (we need it only to be linear
% in at least one variable)
\begin{lemma}[{\cite[Lemma 7]{JKRS09}}]\label{lem:monomial-sum}
  Let $m(x_{1}, \ldots, x_{n})$ be a multilinear monomial. Over a field of
  characteristic other than 2, we have
  \begin{equation}
    \sum_{b \in \{-1, 1\}}m(b) = 0.
  \end{equation}
\end{lemma}

\begin{proof}
  For some $x_{i}$, we can write $m = x_{i} \cdot m'$, where the degree of $x_{i}$
  in $m'$ is 0. Then
  \begin{align*}
    \sum_{b \in \{1, -1\}^{n}}m(b)
    &= \sum_{a \in \{-1, 1\}}\sum_{b' \in \{1, -1\}^{n-1}}a \cdot m'(b') \\
    &= \sum_{a \in \{-1, 1\}} a \cdot \mleft(\sum_{b' \in \{1, -1\}^{n-1}} m'(b')\mright) \\
    &= \mleft(\sum_{b' \in \{1, -1\}^{n-1}} m'(b')\mright) - \mleft(\sum_{b' \in \{1, -1\}^{n-1}} m'(b')\mright) \\
    &= 0.
  \end{align*}
\end{proof}

\section{Statistics}

\begin{defn}\label{def:random-var}\index{random variable}
  A \emph{random variable} is % TODO
\end{defn}

\begin{defn}\label{def:stat-indep}\index{statistical independence}
  Two random variables are \emph{statistically independent} if
\end{defn}

% TODO: Lots more probably needs to go here

% TODO: Talk about "linear independence equals statistical independence" as an intro

\begin{thm}[{\cite[Claim 2]{CFGS22}}]\label{thm:lin-indep-stat-indep}
  Let $\mathbb{F}$ be a finite field and $D$ a finite set. Let
  $V \subseteq \mathbb{F}^{D}$ be a vector space, and let $v$ be a uniform random
  variable over $V$. For any subdomains $S, S' \subseteq D$, the restrictions $v|_{S}$
  and $v|_{S'}$ are statistically dependent if and only if there exist constants
  $c \in \mathbb{F}^{S}$ and $d \in \mathbb{F}^{S'}$ such that
  \begin{enumerate}
    \item There exists $w \in V$ such that $c \cdot w \ne 0$, and
    \item For all $w \in V$, $c \cdot w = d \cdot w$.
  \end{enumerate}
\end{thm}

\begin{proof}
  % TODO
\end{proof}

\chapter{Relativization}

% TODO: Examples

An important prerequisite to understanding algebrization is the similar, but
simpler, concept of \emph{relativization}, also called \emph{oracle separation}.
To do this, we first must define an \emph{oracle}.
\begin{defn}[{\cite[Def.\ 2.1]{AW09}}]\label{def:oracle}\index{oracle}
  An \emph{oracle} $A$ is a collection of Boolean functions
  $A_{m}: \{0, 1\}^{m} \rightarrow \{0, 1\}$, one for each natural number $m$.
\end{defn}
There are several ways to think of an oracle; this will extend the most
naturally when it comes time to define an extension oracle in
\cref{def:ext-oracle}. Another way to think of an oracle is as a subset
$A \subseteq \{0, 1\}^{*}$. This allows us to think of $A$ as a language. Since we can
do this, it gives us the ability to think of the complexity of the oracle. If we
want to think about the subset in terms of our functions, we can write $A$ as
\begin{equation}
  A = \bigcup_{m \in \mathbb{N}}\mleft\{x \in \{0, 1\}^{m} \mid A_{m}(x) = 1\mright\}.
\end{equation}
% FIXME: Will we actually do this?
We will use the Iverson bracket defined in \cref{def:iverson-bracket} for this
purpose: allowing us to think of $A$ as the set and $[A]$ as the function.

\begin{example}\label{ex:oracle-function}
  Let $m = 3$. The function
  \begin{equation}
    \begin{aligned}
      f: \{0, 1\}^{3} &\rightarrow \{0, 1\} \\
      abc &\mapsto b
    \end{aligned}
  \end{equation}
  is an oracle function. We can think of $f$ as corresponding to the set
  $\{010, 011, 110, 111\}$.
\end{example}

\begin{example}\label{ex:oracle-full}
  For each $n \in \mathbb{N}$, define
  \begin{equation}
    \begin{aligned}
      f_{n}: \{0, 1\}^{n} &\rightarrow \{0, 1\} \\
      a_{1}a_{2} \cdots a_{n} &\mapsto a_{n}.
    \end{aligned}
  \end{equation}
  Then the set $\{f_{n}\}$ forms an oracle, whose corresponding language is the
  set of all binary representations of odd numbers.
\end{example}

% TODO: Example for oracle

An oracle is not particularly interesting mathematical object on its own (after
all, it is simply a set of arbitrary Boolean functions); its utility comes from
when it interacts with a Turing machine. A normal Turing machine does not have
the facilities to interact with an oracle, so we need to define a small
extension to a standard Turing machine to allow for this.

\begin{defn}[{\cite[Def.\ 3.6]{AB09}}]\label{def:tm-oracle}\index{Turing machine!with oracle}
  A \emph{Turing machine with an oracle} is a Turing machine with an additional
  tape, called the \emph{oracle tape}, as well as three special states:
  $q_{\text{query}}$, $q_{\text{yes}}$, and $q_{\text{no}}$. Further, each
  machine is associated with an oracle $A$. During the execution of the machine,
  if it ever moves into the state $q_{\text{query}}$, the machine then (in one
  step) takes the output of $A$ on the contents of the oracle tape, moving into
  $q_{\text{yes}}$ if the answer is 1 and $q_{\text{no}}$ if the answer is 0.
\end{defn}

Of course, the question now becomes how we can effectively use an oracle in an
algorithm. The previously-mentioned conception of an oracle as a set of strings
is useful here. If we consider the set of strings as being a \emph{language} in
its own right, then querying the oracle is the same as determining whether a
string is in the langauge, just in one step. If the language is computationally
hard, this means our machine can get a significant power boost from the right
oracle.

\begin{defn}[{\cite[Def.\ 2.1]{AW09}}]\label{def:oracle-class}
  For any complexity class $\mathcal{C}$, the complexity class $\mathcal{C}^{A}$ is the class of all
  languages determinable by a Turing machine with access to $A$ in the number of
  steps defined for $\mathcal{C}$.
\end{defn}

We will be using this definition in many places, so we should take a moment to
look at it in more depth. First, it is important to realize that $\mathcal{C}^{A}$ is a
set of \emph{languages}, not \emph{machines}: despite the notation, augmenting
$\mathcal{C}$ with an oracle does not modify any languages, it just adds new ones that are
computable. Second, since a machine can always ignore its oracle, it follows
that adding an oracle can only increase the number of languages in the class,
never decrease it.

\begin{lemma}\label{thm:relativizing-increases}
  For any complexity class $\mathcal{C}$ and oracle $A$, $\mathcal{C} \subseteq \mathcal{C}^{A}$.
\end{lemma}

\begin{proof}
  Let $L \in \mathcal{C}$ and $M$ be a machine that determines $L$. Then the oracle machine
  $M'$ that simulates $M$ on its input and makes no queries to the oracle will
  also accept exactly $L$. Since $M'$ is a $\mathcal{C}^{A}$ machine for any oracle $A$,
  it follows that $L \in \mathcal{C}^{A}$ and hence $\mathcal{C} \in \mathcal{C}^{A}$.
\end{proof}

While the above lemma tells us that $\mathcal{C} \subseteq \mathcal{C}^{A}$ always, another interesting
question is when $\mathcal{C} = \mathcal{C}^{A}$. We do have a notion for this, called
\emph{lowness}. Lowness can be defined for both individual languages and
complexity classes; we will define both here.

% TODO: Cite all these
\begin{defn}\label{def:low-class}\index{low}
  A language $L$ is \emph{low} for a class $\mathcal{C}$ if $\mathcal{C}^{L} = \mathcal{C}$.
\end{defn}

\begin{defn}\label{def:low-lang}\index{low}
  A complexity class $\mathcal{D}$ is \emph{low} for a class $\mathcal{C}$ if each language in $\mathcal{D}$
  is low for $\mathcal{C}$.
\end{defn}

Of particular interest to us will be classes that are low for \emph{themselves}.
We care about these classes because they can use other problems from the same
class as a subroutine without issue; in particular recursion and iteration both
work here. Thankfully, both $\P$ and $\PSPACE$ are low for themselves (it turns
out $\NP$ is probably not); this allows us to easily write algorithms that
recurse for classes in both of our most common classes.

\begin{thm}\label{thm:p-low}
  $\P$ is low for itself.
\end{thm}

\begin{proof}
  % TODO: Rewrite?
  Let $L \in \P$ and let $K \in \P^{L}$. Let $M(L)$ be the determiner of $L$ and
  $M(K)$ be the determiner of $K$. Further, let $\hat{M}(K)$ be the determiner
  of $K$ but with access to $L$ as an oracle. We aim to show $K \in \P$. Let
  $p_{L}(n)$ be a polynomial upper bound of the runtime of $M(L)$ on an input of
  length $n$, and let $p_{\hat{K}}(n)$ be similar. Since $M(K)$ can call $M(L)$ no
  more than $p_{\hat{K}}(n)$ times, it follows that
  $p_{K}(n) \le p_{\hat{K}}(p_{L}(n))$. Hence, the runtime of $M(K)$ is bounded
  above by a polynomial, and thus $K \in P$.
\end{proof}

\begin{thm}\label{thm:pspace-low}
  $\PSPACE$ is low for itself.
\end{thm}

\begin{proof}
  The proof is very similar to that for \cref{thm:p-low}, but with space instead
  of time. Since memory usage is bounded above by some polynomial, and
  polynomials are closed under composition, it follows that $\PSPACE$ is low for
  itself.
\end{proof}

\section{Defining relativization}

We are now ready to define what relativization is. First, note that
relativization is a statement about a \emph{result}: we talk about inclusions
relativizing, not sets themselves.

% TODO: Cite
\begin{defn}\label{def:relativization}\index{relativization}
  Let $\mathcal{C}$ and $\mathcal{D}$ be complexity classes such that $\mathcal{C} \subseteq \mathcal{D}$. We say the result
  $\mathcal{C} \subseteq \mathcal{D}$ \emph{relativizes} if $\mathcal{C}^{A} \subseteq \mathcal{D}^{A}$ for all oracles $A$. Conversely,
  if there exists $A$ such that $\mathcal{C} \nsubseteq \mathcal{D}$, we say that the result $\mathcal{C} \subseteq \mathcal{D}$
  \emph{does not relativize}.
\end{defn}

\begin{defn}\label{def:relativization-ne}
  Let $\mathcal{C}$ and $\mathcal{D}$ be complexity classes such that $\mathcal{C} \nsubseteq \mathcal{D}$. We say the result
  $\mathcal{C} \nsubseteq \mathcal{D}$ \emph{relativizes} if $\mathcal{C}^{A} \nsubseteq \mathcal{D}^{A}$ for all oracles $A$. Conversely,
  if there exists $A$ such that $\mathcal{C} \subseteq \mathcal{D}$, we say that the result $\mathcal{C} \nsubseteq \mathcal{D}$
  \emph{does not relativize}.
\end{defn}

We start with a very straightforward example of a relativizing result.

\begin{lemma}\label{lem:pa-subset-npa}
  For any oracle $A$, $\P^{A} \subseteq \NP^{A}$. Equivalently, the result $\P \subseteq \NP$
  relativizes.
\end{lemma}

\begin{proof}
  Since any deterministic Turing machine is also a nondeterministic machine, it
  follows that a machine that solves a $\P^{A}$ problem is also an $\NP^{A}$
  machine. Hence, $\P^{A} \subseteq \NP^{A}$.
\end{proof}

This result tells us that not \emph{everything} is weird in the world of
relativization (although we will soon do our best to find all the weird bits):
if we have a machine that can do more operations without an oracle, it can still
do more operations with an oracle. Further, for the question of $\P$ vs.\ $\NP$
that we will discuss in \cref{sec:rel-p-np}, this means that the question we
care about is whether $\NP \subseteq^{?} \P$ relativizes. As such, the question we are
asking simplifies to determining where $\P^{A} = \NP^{A}$ and where
$\P^{A} \subsetneq \NP^{A}$.

Now that we have talked about set inclusions relativizing, we need to define the
other side of the coin: \emph{proofs} can relativize as well as results.
Unfortunately, this needs to be a somewhat informal definition as formally
delineating different types of proof is far beyond the scope of this paper.
However, the definition we offer here will be sufficient for our purposes.

\begin{defn}\label{def:relativizing-result}
  We say a \emph{proof relativizes} if it is not made invalid if the relevant
  classes are replaced with oracle classes, i.e., a proof that $\mathcal{C} \subseteq \mathcal{D}$
  \emph{relativizes} if the same proof can be used to show $\mathcal{C}^{A} \subseteq \mathcal{D}^{A}$ for
  all oracles $A$ with minimal modifications.
\end{defn}

This gives us a reason to care about relativization as a concept: if our proofs
are relativizing then we know not to try to use them to prove nonrelativizing
results. In particular, we will show in \cref{sec:rel-p-np} that the famous $\P$
vs.\ $\NP$ problem will not relativize regardless of the outcome, and then in
\cref{sec:diag-relativizes} we will show that the common proof technique of
diagonalization \emph{does} in fact relativize.

Now that we have given ourselves a reason to care about oracles and how they
interact with Turing machines, we now turn to the question of how a machine can
gain information about the oracle it queries. We will do this with the notion of
\emph{query complexity}.

\section{Query complexity}\label{sec:query-complexity}

The goal of query complexity is to ask questions about some Boolean function
$A: \{0, 1\}^{n} \rightarrow \{0, 1\}$ by querying $A$ itself. For this, we will
interchangeably think of $A$ as a \emph{function} as well as a bit string of
length $N = 2^{n}$, where each string element is $A$ applied to the $i$th string
of length $n$, arranged in some lexicographical order. % TODO: Better way to phrase this
We can further think of the property itself as being a Boolean function; a
function that takes as input the bit-string representation of $A$ and outputs
whether or not $A$ has the given property. We will call the function
representing the property $f$. When viewed like this, $f$ is a function from
$\{0, 1\}^{N}$ to $\{0, 1\}$. We define three types of query complexity for
three of the most common types of computing paradigms: deterministic,
randomized, and quantum. Nondeterministic query complexity is interesting, but
it is outside the scope of this paper.
% TODO: Why on earth does this paper not define nondeterministic query complexity?

% TODO: Find better source for these definitions
% Perhaps rephrase in the style of AW09 Def. 4.1?
\begin{defn}[{\cite[17]{AW09}}]\label{def:det-qc}\index{query complexity!deterministic}
  Let $f: \{0, 1\}^{N} \rightarrow \{0, 1\}$ be a Boolean function. Then the
  \emph{deterministic query complexity} of $f$, which we write $D(f)$, is the
  minimum number of queries made by any deterministic algorithm with access to
  an oracle $A$ that determines the value of $f(A)$.
  % TODO: I don't quite understand the phrasing here; perhaps rephrase
\end{defn}

To make this more clear, let us give an example problem.

\begin{defn}\label{def:or-problem}\index{OR@$\OR$}
  The $\OR$ problem is the following oracle problem:
  \begin{quote}
    Let $A: \{0, 1\}^{n} \rightarrow \{0, 1\}$ be an oracle. The function $\OR(A)$ returns
    1 if there exists a string on which $A$ returns 1, and $0$ otherwise.
  \end{quote}
\end{defn}

The question is then what the deterministic query complexity of the $\OR$
function is.

\begin{thm}
  The $\OR$ problem has a deterministic query complexity of $2^{n}$.
\end{thm}

\begin{proof}
  First, note that any algorithm that determines the $\OR$ problem can stop as
  soon as it queries $A$ and gets an output of $1$. Hence, for any algorithm
  $M$, let $\{s_{i}\}$ be the sequence of queries $M$ makes to $A$ on the
  assumption that it always recieves a response of $0$. If
  $\abs{\{s_{i}\}} \le 2^{n}$, there exists some $s \in \{0, 1\}^{n}$ not queried.
  In that case, $M$ will not be able to distinguish the zero oracle from the
  oracle that outputs $1$ only when given $s$. Hence, $M$ must query every
  string of length $n$ and thus the query complexity is $2^{n}$.
\end{proof}

From this, we get that the $\OR$ problem cannot be solved any better than by
enumerative checking. This makes intuitive sense because none of the results we
get by querying $A$ imply anything about what $A$ will do on other values, since
$A$ can be an arbitrary function. Later on (in \cref{sec:alg-query-complexity}),
we will look at what happens when we give ourselves access to a
\emph{polynomial}, where querying one point could tell us information about
others.

For the next two definitions, since their Turing machines include some element
of randomness, we only require that they succeed with a $2/3$ probability. This
is in line with most definitions of complexity classes involving random
computers.

\begin{defn}[{\cite[17]{AW09}}]\label{def:rand-qc}\index{query complexity!randomized}
  Let $f: \{0, 1\}^{N} \rightarrow \{0, 1\}$ be a Boolean function. Then the
  \emph{randomized query complexity} of $f$, which we write $D(f)$, is the
  minimum number of queries made by any randomized algorithm with access to an
  oracle $A$ that evaluates $f(A)$ with probability at least $2/3$.
\end{defn}

% TODO: Talk about how quantum oracles are weird?

\begin{defn}[{\cite[17]{AW09}}]\label{def:quant-qc}\index{query complexity!quantum}
  Let $f: \{0, 1\}^{N} \rightarrow \{0, 1\}$ be a Boolean function. Then the
  \emph{quantum query complexity} of $f$, which we write $D(f)$, is the
  minimum number of queries made by any quantum algorithm with access to an
  oracle $A$ that evaluates $f(A)$ with probability at least $2/3$.
\end{defn}
% TODO: Examples

\section{Relativization of $\P$ vs.\ $\NP$}\label{sec:rel-p-np}

% TODO? move to relativization section as an example?
An important example of relativization is that of $\P$ and $\NP$. While the
question of if $\P = \NP$ is still open, we aim to show that \emph{regardless of
the answer}, the result does not algebrize. To do this, we show that there are
some oracles $A$ where $\P^{A} = \NP^{A}$, and some where $\P^{A} \ne \NP^{A}$.

Additionally, it should be noted that the similarity of relativization to
algebrization means that the structure of these proofs will return in
\cref{sec:alg-p-np} when we show the algebrization of $\P$ and $\NP$.

\subsection{Equality}

The more straightforward of the two proofs is the oracle where
$\P^{A} = \NP^{A}$, so we shall begin with that.

\begin{thm}[{\cite[Theorem 2]{BGS75}}]\label{thm:p-np-rel}
  There exists an oracle $A$ such that $\P^{A} = \NP^{A}$.
\end{thm}

\begin{proof}
  For this, we can let $A$ be any $\PSPACE$-complete language. By letting our
  machine in $\P$ be the reducer from $A$ to any other language in $\PSPACE$, we
  therefore get that $\PSPACE \subseteq \P^{A}$. Similarly, if we have a problem in
  $\NP^{A}$, we can verify it in polynomial space without talking to $A$ at all
  (by having our machine include a determiner for $A$). Hence, we have that
  $\NP^{A} \subseteq \NPSPACE$. Further, a celebrated result of Savitch~\cite{Sav70}
  (which we briefly discussed as \cref{thm:savitch}) is that
  $\PSPACE = \NPSPACE$. Combining all these results, we get the chain
  \begin{equation}
    \NP^{A} \subseteq \NPSPACE = \PSPACE \subseteq \P^{A} \subseteq \NP^{A}.
  \end{equation}
  This is a circular chain of subset relations, which means everything in the
  chain must be equal. Hence, $\P^{A} = \NP^{A} = \PSPACE$.
\end{proof}

For a slightly more intuitive view of what this proof is doing, what we have
done is found an oracle that is so powerful that it dwarfs any amount of
computation our actual Turing machine can do. Hence, the power of our machine is
really just the same as the power of our oracle, and since we have given both
the $\P$ and $\NP$ machine the same oracle, they have the same power.

\subsection{Inequality}

Having shown that an oracle exists where $\P^{A} = \NP^{A}$, we now endeavor to
find one where $\P^{A} \ne \NP^{A}$. This piece of the proof is less simple than
the previous section, and it uses a diagonalization argument to construct the
oracle. Before we dive in to the main proof, however, we need to define a few
preliminaries.

\begin{defn}[{\cite[436]{BGS75}}]\label{def:l(x)}\index{L(X)@$L(X)$}
  Let $X$ be an oracle. The language $L(X)$ is the set
  \begin{equation*}
    L(X) = \{x \mid \text{there is } y \in X \text{ such that } \abs{y} = \abs{x}\}.
  \end{equation*}
\end{defn}

\begin{example}\label{ex:l(x)-simple}
  Consider the language $X = \{0, 11, 0100\}$. The language $L(X)$ is the
  language consisting of all strings of length $1$, $2$, and $4$.
\end{example}

Our eventual goal will be to construct a language $X$ such that
$L(X) \in \NP^{X} \setminus \P^{X}$. Of particular note is that we can rather nicely put a
upper bound on the complexity of $L(X)$ when given $X$ as an oracle, regardless
of the value of $X$. This fact is what gives us the freedom to construct $X$ in
such a way that $L(X)$ will not be in $\P^{X}$.

\begin{lemma}[{\cite[436]{BGS75}}]\label{lem:l(x)-in-np}
  For any oracle $X$, $L(X) \in \NP^{X}$.
\end{lemma}

\begin{proof}
  Let $S$ be a string of length $n$. If $S \in L(X)$, then a witness for $S$ is
  any string $S'$ such that $\abs{S} = \abs{S'}$ and $S' \in X$. Since a machine
  with query access to $X$ can query whether $S'$ is in $X$ in one step, it
  follows that we can verify that $S \in L(X)$ in polynomial time.
\end{proof}

With this lemma as a base, we can now move on to our main theorem.

\begin{thm}[{\cite[Theorem 3]{BGS75}}]\label{thm:p-np-nrel}
  There exists an oracle $A$ such that $\P^{A} \ne \NP^{A}$.
\end{thm}

\begin{proof}
  Our goal is to construct a set $B$ such that $L(B) \notin \P^{B}$. We shall
  construct $B$ in an interative manner. We do this by taking a sequence
  $\{P_{i}\}$ of all machines that recongize some language in $\P^{A}$, and then
  constructing $B$ such that for each machine in the sequence, there is some
  part of $L(B)$ it cannot recognize. This technique is called
  \emph{diagonalization}, and it is used in many places in computer science
  theory.\footnote{This argument style is named after \emph{Cantor's diagonal
      argument}, which was originally used to prove that the real numbers are
    uncountable~\cite[Thm. 2.14]{Ru76}.} Additionally, we define $p_{i}(n)$ to
  be the maximum running time of $P_{i}$ on an input of length $n$. We give the
  following algorithm to construct $B$:

  \begin{algorithm}[H]
    % FIXME: Do the P_i need to be P machines? Everybody is unclear on this
    \KwIn{A sequence of $\P$ oracle machines $\{P_{i}\}_{i=1}^{\infty}$}
    \KwOut{A set $B$ such that $L(B) \notin \P^{B}$}
    % TODO: Define p_i(n)
    $B(0) \leftarrow \varnothing$\;
    $n_{0} \leftarrow 0$\;
    \For{$i$ starting at $1$}{
      Let $n > n_{i}$ be large enough that
      $p_{i}(n) < 2^{n}$\;\nllabel{line:def-n}
      Run $P_{i}^{B(i-1)}$ on input $0^{n}$\;\nllabel{line:computation}
      \If{$P_{i}^{B(i-1)}$ rejects $0^{n}$}{
        Let $x$ be a string of length $n$ not queried during the above
        computation\;\nllabel{line:not-queried}
        $B(i) \leftarrow B(i-1) \sqcup \{x\}$\;
      }
      $n_{i+1} \leftarrow 2^{n}$\;
    }
    $B \leftarrow \bigcup_{i}B(i)$\;
    \caption{An algorithm for constructing $B$}\label{alg:construct-b}
  \end{algorithm}

  Now that we have presented the algorithm, let us demonstrate its soundness.
  First, note that since $P_{i}$ runs in polynomial time, $p_{i}(n)$ is bounded
  above by a polynomial, and hence there will always exist an $n$ as defined in
  line~\ref{line:def-n}. Next, since there are $2^{n}$ strings of length $n$ and
  since $p_{i}(n) < 2^{n}$, we know that there must be some $x$ to make
  line~\ref{line:not-queried} well-defined. While our algorithm allows $x$ to be
  any string, if it is necessary to be explicit in which we choose, then picking
  $x$ to be the smallest string in lexicographic order is a standard choice.

  We should also briefly mention that this algorithm does not terminate. This is
  okay because we are only using it to construct the set $B$, which does not
  need to be bounded. If this were to be made practical, since the sequence of
  $n_{i}$s is monotonically increasing, the set could be constructed ``lazily''
  on each query by only running the algorithm until $n_{i}$ is greater than the
  length of the query.

  % FIXME: Should this "end goal" section be moved to before the algorithm?
  Next, we demonstrate that $L(B) \notin \P^{B}$. The end goal of our instruction is
  a set $B$ such that if $P_{i}^{B}$ accepts $0^{n}$ then there are no strings
  of length $n$ in $B$, and if $P_{i}^{B}$ rejects, then there is a string of
  length $n$ in $B$. This means that no $P_{i}$ accepts $L(B)$, and hence
  $L(B) \notin \NP^{B}$.

  The central idea behind the proper functioning of our algorithm is that adding
  strings to our oracle \emph{cannot change the output if they are not queried}.
  This is what we do in line~\ref{line:def-n}: we need our input length to be
  long enough to guarantee that a non-queried string exists. Since the number of
  queried strings is no greater than $p_{i}(n)$, and there are $2^{n}$ strings
  of length $n$, there must be some string not queried.

  Next, we run $P_{i}^{B(i-1)}$ on all the strings we have already added. If it
  accepts, then we want to make sure that no string of length $n$ is in $B$;
  that is, $0^{n}$ is not in $L(B)$. Hence, in this particular loop we add
  nothing to $B(i)$. If $P_{i}^{B(i-1)}$ rejects, we then need to make sure that
  $0^{n} \in L(B)$ but in a way that does not affect the output of
  $P_{i}^{B(i-1)}$. Hence, we find a string that $P_{i}^{B(i-1)}$ did not query
  (and thus will not affect the result) and add it to $B(i)$.

  Having done this, we then set $n_{i+1}$ to be $2^{n}$. Since
  $p_{i}(n) < 2^{n}$, it follows that no previous machine could have queried any
  strings of length $n_{i+1}$.\footnote{A word of caution: we only care about
    what $P_{i}$ does on input $n_{i}$, \emph{not any other input}. This is
    because we only need each machine to be incorrect for some $i$, not all
    $i$.} This way, we ensure our previous machines do not accidentally have
  their output change due to us adding a string they queried.

  % TODO: Note about how it's fine that this doesn't actually halt b/c it's just
  % in the construction of the set
  Having run this over all polynomial-time Turing machines, we have a set $L(B)$
  such that no machine in $\P^{B}$ accepts it, which tells us $L(B) \notin \P^{B}$.
  But, \cref{lem:l(x)-in-np} already told us $L(B) \in \NP^{B}$. Hence,
  $\P^{B} \ne \NP^{B}$.
\end{proof}

\section{Diagonalization relativizes}\label{sec:diag-relativizes}

Of course, determining that $\P$ vs $\NP$ does not relativize is only important
if the proof techniques used in practice \emph{do} in fact relativize. Rather
unfortunately, it turns out that simple diagonalization is a relativizing
result.

% FIXME: Are there formal definitions of diagonalization?
While diagonalization itself does not have a formal definition, we can still
think about it informally. Looking at our construction of $B$, which we did
using diagonalization, notice that our definition never really cared about how
the $P_{i}$ worked, just about the results it produced. Hence, if it were to be
possible to modify \cref{alg:construct-b} to construct $B \in \NP \setminus \P$, the proof
would remain the same if we were to replace our sequence $\{P_{i}\}$ with a
sequence of machines in $\P^{A}$ for some $\PSPACE$-complete $A$. However, this
would lead to a contradiction, as we showed in \cref{thm:p-np-rel} that in that
case, $\P^{A} = \NP^{A}$! This tells us that a simple diagonalization argument
would not suffice to determine separation between $\P$ and $\NP$.

\section{Arithmetization does not relativize}\label{sec:arith-non-rel}

% TODO

\chapter{Algebrization}\label{chap:algebrization}

Algebrization, originally described by Aaronson and Wigderson~\cite{AW09}, is an
extension of relativization. While relativization deals with oracles that are
Boolean functions, algebrization extends oracles to be a collection of
polynomials over finite fields. Since any field contains the set $\{0, 1\}$, we
can think about our new oracles as \emph{extending} some specific oracle $A$, so
that both oracles agree on the set $\{0, 1\}^{n} \subseteq \mathbb{F}^{n}$. We formalize
this notion below.

\begin{defn}[{\cite[Def.\ 2.2]{AW09}}]\label{def:ext-oracle}\index{extension oracle}
  % TODO: Reuse definition of extension polynomial from earlier
  Let $A_{m}: \{0, 1\}^{m} \rightarrow \{0, 1\}$ be a Boolean function and let
  $\mathbb{F}$ be a finite field. Then an \emph{extension} of $A_{m}$ over
  $\mathbb{F}$ is a polynomial
  $\tilde{A}_{m,\mathbb{F}}: \mathbb{F}^{m} \rightarrow \mathbb{F}$ such that
  $\tilde{A}_{m,\mathbb{F}}(x) = A_{m}(x)$ whevever $x \in \{0, 1\}^{m}$. Also,
  given an oracle $A = (A_{m})$, an \emph{extension} $\tilde{A}$ of $A$ is a
  collection of polynomials
  $\tilde{A}_{m,\mathbb{F}}: \mathbb{F}^{m} \rightarrow \mathbb{F}$, one for each positive
  integer $m$ and finite field $\mathbb{F}$, such that
  \begin{enumerate}
    \item $\tilde{A}_{m,\mathbb{F}}$ is an extension of $A_{m}$ for all
          $m,\mathbb{F}$, and
    \item there exists a constant $c$ such that
          $\mdeg(\tilde{A}_{m,\mathbb{F}}) \le c$ for all $m, \mathbb{F}$.
          % TODO: Rephrase point 2 in terms of F[x_{1,...,n}^{<= c}]
  \end{enumerate}
\end{defn}

Take note that an oracle can have many different extension oracles, since one
can construct an infinite number of polynomials that go through a set of points.
For this reason, when dealing with oracles in practice, we will also often be
interested in oracles of a particular multidegree, which limits our options for
oracles in potentially-interesting ways.

\begin{example}\label{ex:oracle-function-ext}
  Consider the function we defined in \cref{ex:oracle-function}:
  \begin{equation}
    \begin{aligned}
      f: \{0, 1\}^{3} &\rightarrow \{0, 1\} \\
      abc &\mapsto b.
    \end{aligned}
  \end{equation}
  An extension of that function is the polynomial
  \begin{equation}
    \begin{aligned}
      \tilde{f}: \mathbb{F}^{3} &\rightarrow \mathbb{F}^{3} \\
      (a,b,c) &\mapsto b.
    \end{aligned}
  \end{equation}
  While this is a relatively trivial polynomial, there are more non-trivial
  ones, for example
  \begin{equation}
    \begin{aligned}
      \tilde{f}: \mathbb{F}^{3} &\rightarrow \mathbb{F}^{3} \\
      (a,b,c) &\mapsto a^{3}c^{3} + b^{2} - ac.
    \end{aligned}
  \end{equation}
  Notice that on $\{0, 1\}$, $x^{2} = x$, which allows us to see that
  $\tilde{f}$ is a valid extension of $f$.
\end{example}

% TODO: Example of an extension oracle

\begin{defn}[{\cite[Def.\ 2.2]{AW09}}]\label{def:ext-oracle-class}
  For any complexity class $\mathcal{C}$ and extension oracle $\tilde{A}$, the complexity
  class $\mathcal{C}^{\tilde{A}}$ is the class of all languages determinable by a Turing
  machine with access to $\tilde{A}$ with the requirements for $\mathcal{C}$.
\end{defn}

Next, we need to formally define what algebrization is.

\begin{defn}[{\cite[Def.\ 2.3]{AW09}}]\label{def:algebrization}\index{algebrization}
  Let $\mathcal{C}$ and $\mathcal{D}$ be complexity classes such that $\mathcal{C} \subseteq \mathcal{D}$. We say the result
  $\mathcal{C} \subseteq \mathcal{D}$ \emph{algebrizes} if $\mathcal{C}^{A} \subseteq \mathcal{D}^{\tilde{A}}$ for all oracles $A$ and
  finite field extensions $\tilde{A}$ of $A$. Conversely, if there exists $A$
  and $\tilde{A}$ such that $\mathcal{C} \nsubseteq \mathcal{D}$, we say that the result $\mathcal{C} \subseteq \mathcal{D}$ \emph{does
    not algebrize}.
\end{defn}

\begin{defn}[{\cite[Def.\ 2.3]{AW09}}]\label{def:algebrization-neq}
  Let $\mathcal{C}$ and $\mathcal{D}$ be complexity classes such that $\mathcal{C} \nsubseteq \mathcal{D}$. We say the result
  $\mathcal{C} \nsubseteq \mathcal{D}$ \emph{algebrizes} if $\mathcal{C}^{A} \nsubseteq \mathcal{D}^{\tilde{A}}$ for all oracles $A$ and
  finite field extensions $\tilde{A}$ of $A$. Conversely, if there exists $A$
  and $\tilde{A}$ such that $\mathcal{C} \subseteq \mathcal{D}$, we say that the result $\mathcal{C} \nsubseteq \mathcal{D}$ \emph{does
    not algebrize}.
\end{defn}

\section{Algebraic query complexity}\label{sec:alg-query-complexity}

Similarly to how we defined query complexity in \cref{sec:query-complexity}, our
notion of algebrization requires a definition of \emph{algebraic} query
complexity. % TODO: More (connect to previous section)

\begin{defn}[{\cite[Def. 4.1]{AW09}}]\label{def:aqc}\index{query complexity!algebraic}
  Let $f: \{0, 1\}^{N} \rightarrow \{0, 1\}$ be a Boolean function, $\mathbb{F}$ be a
  field, and $c$ be a positive integer. Also, let $\mathbb{M}$ be the set of
  deterministic algorithms $M$ such that $M^{\tilde{A}}$ outputs $f(A)$ for
  every oracle $A: \{0, 1\}^{n} \rightarrow \{0, 1\}$ and every finite field extension
  $\tilde{A}: \mathbb{F}^{n} \rightarrow \mathbb{F}$ of $A$ with $\mdeg(\tilde{A}) \le c$.
  Then, the deterministic algebraic query complexity of $f$ over $\mathbb{F}$ is
  defined as
  \begin{equation}
    \tilde{D}_{\mathbb{F}, c}(f) = \min_{M \in \mathcal{M}}\mleft(
      \max_{A, \tilde{A}: \mdeg(\tilde{A}) \le c}T_{M}(\tilde{A})
    \mright),
  \end{equation}
  where $T_{M}(\tilde{A})$ is the number of queries to $\tilde{A}$ made by
  $M^{\tilde{A}}$.
  % TODO: Can this be made more intelligible?
\end{defn}

Our goal here is to find the \emph{worst}-case scenario for the \emph{best}
algorithm that calculates the property $f$. The difference between this and
\cref{def:det-qc} is twofold: first, our algorithm $M$ has access to
an extension oracle of $A$, and second, that we can limit our $\tilde{A}$ in
its maximum multidegree. For the most part, we will focus on equations with
multidegree 2, which is enough to get the results we want.

As an example, let us look at the same $\OR$ problem we defined in
\cref{def:or-problem}.

% FIXME:
\begin{thm}[{\cite[Thm.\ 4.4]{AW09}}]\label{thm:or-algebraic}
  $\tilde{D}_{\mathbb{F},2}(\OR) = 2^{n}$ for every field $\mathbb{F}$.
\end{thm}

\begin{proof}
  First note that $2^{n}$ is an upper bound for the number of queries necessary
  since we can query every point in $\{0, 1\}^{n}$, of which there are $2^{n}$.

  Let $M$ be a deterministic algorithm and let $\mathcal{Y}$ be the set of points queried
  by $M$ in the case where $M$ always recieves $0$ as a response. So long as
  $\abs{\mathcal{Y}} < 2^{n}$, there exists by \cref{thm:multiquad-extension} a
  multiquadratic extension polynomial $\tilde{A}: \mathbb{F}^{n} \rightarrow \mathbb{F}$
  such that $\tilde{A}(y) = 0$ for all $y \in \mathcal{Y}$ but $\tilde{A}(w) = 1$ for some
  $w \in \{0, 1\}^{n}$. As such, if $M$ queries less than $2^{n}$ points then it
  would not be able to tell the difference between $\tilde{A}$ and the zero
  function. However, $\OR(A) = 1$ and $\OR(0) = 0$, so it would get the
  incorrect answer for one of them. Hence if $M$ queries fewer than $2^{n}$
  points it cannot solve the $\OR$ problem.
\end{proof}

Note that this works even if $M$ is adaptive: if $M$ ever recieves a nonzero
response it (correctly) knows $\OR(A) = 1$, so it can accept immediately. As
such, we know that any contradiction must come when $M$ has only ever seen zeros
as responses.

This gives us a potentially counterintuitive property of algebraic query
complexity: while it would seem that giving our machine a polynomial (and a
polynomial of multidegree only 2, at that) would give us the ability to solve
the hardest problems more quickly, that turns out not to be the case.

Now, while this is true for polynomials of multidegree 2, it turns out that if
we restrict our oracles to being simply \emph{multilinear} polynomials, we do
get a speedup.

\begin{thm}[{\cite[Thm. 3]{JKRS09}}]\label{thm:or-multilinear}
  $\tilde{D}_{\mathbb{F},1}(\OR) = 1$ for every field $\mathbb{F}$ with
  characteristic not equal to $2$.
\end{thm}

\begin{proof}
  % TODO
  Let $A: \{0, 1\}^{n} \rightarrow \{0, 1\}$ and $\tilde{A}$ be our extension polynomial.
  Consider the value of $p(1/2, \ldots, 1/2)$. We aim to show that this value is
  equal to $0$ if and only if $A$ is the zero oracle.

  Consider the function
  \begin{equation}
    p'(x_{1}, \ldots, x_{n}) = p(1 - 2x_{1}, \ldots, 1 - 2x_{n}).
  \end{equation}
  Since $1 - 2x$ is a linear polynomial, it follows that $p'$ is itself a
  multilinear polynomial. Further, since the sum over $\{1, -1\}^{n}$ of a
  non-constant multilinear monomial is 0 as per \cref{lem:monomial-sum}, it
  follows that
  \begin{equation}
    \sum_{b \in \{-1, 1\}^{n}}p'(b) = p'(0, \ldots, 0),
  \end{equation}
  i.e., the constant term of $p'$. Further, from our definition of $p'$, we have
  that $p'(0, \ldots, 0) = p(1/2, \ldots, 1/2)$. Hence, we have
  \begin{equation}
    \sum_{b \in \{0, 1\}^{n}}p(b) = p(1/2, \ldots, 1/2).
  \end{equation}
  Since $p(b) \ge 0$ for all $b \in \{0, 1\}^{n}$, it follows that $p(1/2, \ldots, 1/2)$
  is 0 if and only if $p(b) = 0$ for all $b \in \{0, 1\}^{n}$, i.e. exactly when
  $A$ is the zero function.
\end{proof}

% TODO: Deterministic & nondeterministic AQC

\section{Algebrization of $\P$ vs.\ $\NP$}\label{sec:alg-p-np}

As with relativization, an important application of algebrization is in regards
to the $\P$ vs.\ $\NP$ problem.

\begin{defn}[{\cite[Def.\ 6.1]{BFL90}}]\label{def:pspace-robust}\index{PSPACE-robust@$\PSPACE$-robust}
  A language $L$ is \emph{$\PSPACE$-robust} if $\P^{L} = \PSPACE^{L}$.
\end{defn}

% TODO: Move up above thm:p-np-rel so I can reference it there?
\begin{lemma}\label{lem:complete-is-robust}
  Any $\PSPACE$-complete language is also $\PSPACE$-robust.
\end{lemma}

\begin{proof}
  First, we know from \cref{thm:relativizing-increases} that
  $\P^{L} \subseteq \PSPACE^{L}$. Next, let $M \in \PSPACE^{L}$, and we aim to show
  $M \in \P^{L}$. Since $L \in \PSPACE$ and $\PSPACE$ is low for itself, we know
  $M \in \PSPACE$. As such, we know there is a polynomial-time reduction $f$ from
  $M$ to $L$. Hence, we can compute $M$ by running $f$ on the input and then
  testing if that output is in $L$ (using the oracle). Hence, $M \in \P^{L}$ and
  thus $\P^{L} = \PSPACE^{L}$.
\end{proof}

\begin{lemma}[{\cite[Lemma 6.2]{BFL90}}]\label{lem:multilinear-is-pspace}
  Let $L$ be a $\PSPACE$-robust language, with corresponding oracle $A$. Let
  $\tilde{A}$ be the unique multilinear extension oracle of $A$. Then the
  language
  \begin{equation}
    \tilde{L} = \bigcup_{n \in \mathbb{N}}\{(x_{1}, \ldots, x_{n}, z) \in \mathbb{F}^{n+1} \mid \tilde{A}(x_{1}, \ldots, x_{n}) = z\}
  \end{equation}
  is polynomially-equivalent to $L$; that is, $\tilde{L} \in \P^{L}$ and
  $L \in \P^{\tilde{L}}$.
\end{lemma}

The proof of this statement originally given in~\cite{BFL90} has some apparent
problems; we discuss these more thoroughly later on in \cref{app:bug-in-pspace}.
Instead, we present our own proof of the above lemma.

\begin{proof}
  First, we provide a polynomial-time reduction from $L$ to $\tilde{L}$. Since
  for all $x \in \{0, 1\}^{n}$, $\tilde{A}(x) = 1$ if and only if $x \in L$, it
  follows that
  \begin{equation}
    \begin{aligned}
      f: \Sigma^{*} &\rightarrow \Sigma^{*} \\
      x &\mapsto (x, 1)
    \end{aligned}
  \end{equation}
  is a polynomial-time reduction from $L$ to $L'$.

  \begin{algorithm}[H]
    \KwIn{$(x_{1}, \ldots, x_{n}, z) \in \mathbb{F}^{n+1}$}
    \KwOut{Whether $\tilde{A}(x_{1}, \ldots, x_{n}) = z$}
    $z' \leftarrow 0$\;
    \For{$k \in \{0, 1\}^{n}$}{\nllabel{line:for-z-prime}
      Simulate $L$ on input $k$\;
      \If{$k \in L$}{
        \tcp{Compute $d_{k}(x)$}
        $d \leftarrow 1$\;
        \For{$i$ from $1$ to $n$}{\nllabel{line:for-d}
          \eIf{$k_{i} = 1$}{
            $d \leftarrow d \cdot x_{i}$\;
          }{
            $d \leftarrow d \cdot (1 - x_{i})$\;
          }
        }
        $z' \leftarrow z' + d$\;
      }
    }
    \Return{whether $z = z'$}\;
    \caption{Determiner for $\tilde{L}$}\label{alg:l-tilde-det}
  \end{algorithm}

  This algorithm simply calculates the value of $\tilde{A}(x_{1}, \ldots, x_{n})$
  directly, from the explicit definition we gave in
  \cref{cor:low-degree-boolean}, and then compares it to the value of $z$.

  First, we demonstrate that the above algorithm runs in $\P^{L}$. From the
  definition of $\PSPACE$-robustness, we know that we only need to show that the
  algorithm runs in $\PSPACE^{L}$, a much weaker bound. The inner for-loop runs
  in polynomial \emph{time}, hence it must run in polynomial space. The outer
  for-loop runs for $2^{n}$ iterations, so determining that it is in $\P^{L}$ is
  non-trivial. Beyond the inner loop (which we have already discussed), the only
  thing we do in the outer loop is simulate $L$, which can be done in one step
  with access to an oracle for $L$.

  The only memory we need to simulate this oracle (beyond that for the input) is
  space for $d$ and $z'$. We have already shown $d$ needs polynomial space, so
  what remains is $z'$. Since $A(x_{1}, \ldots, x_{n}) \in \{0, 1\}$, each term in the
  sum in \cref{eqn:low-deg-ext-small} is bounded above by $\delta_{\beta}(x)$. This means
  that the value of $z'$ that we compute is bounded above by
  \begin{equation}
    2^{n}\max_{k \in \{0, 1\}^{n}}\delta_{k}(x).
  \end{equation}
  Since each $\delta_{k}(x)$ can be written in polynomial space, and $2^{n}$ can be
  \emph{written} in polynomial space, it follows that $z'$ can as well. Hence,
  \cref{alg:l-tilde-det} is in $\PSPACE^{L}$, and thus is in $\P^{L}$.

  Next, we show that \cref{alg:l-tilde-det} determines $\tilde{L}$. As mentioned
  earlier, our algorithm computes $\tilde{A}$ directly through the equations
  given in \cref{cor:low-degree-boolean}. First, we show the inner loop
  (beginning on line~\ref{line:for-d}) computes $\delta_{k}(x)$. We compute $\delta$
  directly, through the formula described at \cref{eqn:delta-poly-small}. We do
  this by simply iterating through each $i$ and then multiplying $d$ by either
  $x_{i}$ or $1-x_{i}$, as appropriate.

  Second, in this case \cref{eqn:low-deg-ext-small} simplifies to
  \begin{equation}
    \tilde{A}_{n}(x_{1}, \ldots, x_{n}) = \sum_{\beta \in L}\delta_{\beta}(x_{1}, \ldots, x_{n}).
  \end{equation}
  This is exactly what our outer loop does: computes the sum directly through
  iteration.
  Hence, the only thing the above algorithm does is calculate
  $\tilde{A}_{n}(x_{1}, \ldots, x_{n})$ and then compares it to the value we were
  given. As such, it determines $\tilde{L}$.

  Since there is a reduction from $L$ to $\tilde{L}$, we know that $L$ is no
  harder than $\tilde{L}$, and \cref{alg:l-tilde-det} demonstrates that
  $\tilde{L} \in \PSPACE$. Hence, $\tilde{L}$ is $\PSPACE$-complete.
\end{proof}

With that as a base, we can now move on to the main theorem. As before, the more
straightforward proof is the oracle where $\P^{\tilde{A}} = \NP^{A}$, so we
begin with that.

\begin{thm}[{\cite[Theorem 5.1]{AW09}}]\label{thm:p-np-alg}
  There exist $A$, $\tilde{A}$ such that $\NP^{A} = \P^{\tilde{A}}$.
\end{thm}

\begin{proof}
  For this theorem, we use the same technique we did in our proof of
  \cref{thm:p-np-rel}: find a $\PSPACE$-complete language $A$ and work from
  there. If we let $\tilde{A}$ be the unique multilinear extension of $A$,
  \cref{lem:multilinear-is-pspace} tells us $\tilde{A}$ is $\PSPACE$-complete.
  Hence, as mentioned before, we have
  $\NP^{\tilde{A}} \subseteq \NP^{\PSPACE} \subseteq \NPSPACE$, and since $\NPSPACE = \PSPACE$
  and we know from \cref{thm:p-np-rel} that $\PSPACE \subseteq \P^{A}$, it follows
  \begin{equation*}
    \NP^{\tilde{A}} = \NP^{\PSPACE} = \PSPACE = \P^{A}.
  \end{equation*}
\end{proof}

Now it is time for the other case.

\begin{thm}[{\cite[Theorem 5.3]{AW09}}]\label{thm:p-np-nalg}
  There exist $A$, $\tilde{A}$ such that $\NP^{A} \ne \P^{\tilde{A}}$.
\end{thm}

\begin{proof}
  Like in \cref{thm:p-np-nrel}, we aim to ``diagonalize'': iterate over all
  $\P^{\tilde{A}}$ machines to construct a language that none of them can
  recognize. Also like before, we will do this by constructing an oracle
  extension $\tilde{A}$ such that $L(A) \notin \P^{\tilde{A}}$. Since we only give an
  algebraic extension to $\P$ and not $\NP$, we can resuse the result from
  \cref{lem:l(x)-in-np} that $L(A) \in \NP^{A}$. We shall construct $\tilde{A}$
  using the following algorithm:
  \begin{algorithm}[H]
    \KwIn{A sequence of $\P$ oracle machines $\{P_{i}\}_{i=1}^{\infty}$}
    \KwOut{An extension oracle $\tilde{A}$ such that $L(A) \notin \P^{\tilde{A}}$}
    $\tilde{A} \leftarrow \varnothing$\;
    $n_{0} \leftarrow 0$\;
    \For{$i$ starting at $1$}{
      Let $n > n_{i}$ be large enough that
      $p_{i}(n) < 2^{n}$\;
      Run $P_{i}^{\tilde{A}}$ on input $0^{n}$\;
      \eIf{$P_{i}^{B(i-1)}$ rejects $0^{n}$}{
        Let $\mathcal{Y}_{\mathbb{F}}$ be the set of all $y \in \mathbb{F}^{n_{i}}$ queried
        during the above computation\;
        \tcp{See \cref{lem:multiquad-adversary} for why we can do this}
        Let $w \in \{0, 1\}^{n}$ such that the following
        works\;\nllabel{line:def-w}
        \For{all $\mathbb{F}$}{
          Set $\tilde{A}_{n_{i},\mathbb{F}}$ to be a multiquadratic polynomial
          such that $\tilde{A}_{n_{i},\mathbb{F}}(w) = 1$ and
          $\tilde{A}_{n_{i},\mathbb{F}}(y) = 0$ for all
          $y \in \mathcal{Y}_{\mathbb{F}} \cup (\{0, 1\}^{n_{i}} \setminus \{w\})$\;\nllabel{line:set-a}
          % FIXME: Be more clear about what this means
        }
      }{
        Set $\tilde{A}_{n_{i},\mathbb{F}} = 0$ for all $\mathbb{F}$\;
      }
      $n_{i+1} \leftarrow 2^{n}$\;
    }
    $B \leftarrow \bigcup_{i}B(i)$\;
    \caption{An algorithm for constructing $\tilde{A}$}\label{alg:construct-a-tilde}
  \end{algorithm}
  As before, we will start by demonstrating soundness and then move on to why
  the constructed oracle provides the separation we seek.

  Perhaps the least intuitive section of the above algorithm is the section
  beginning at line~\ref{line:def-w}. We want to leverage
  \cref{lem:multiquad-adversary} to show that such a solution exists. We know
  that $p_{i}(n) < 2^{n}$, and since $p_{i}(n)$ is an upper bound on the number
  of total queries, this tells us that there is at least one
  $w \in \{0, 1\}^{n_{i}}$ not queried. From the definition of $\mathcal{Y}_{\mathbb{F}}$,
  we also therefore know that $\sum_{\mathbb{F}}\mathcal{Y}_{\mathbb{F}} < 2^{n}$. Further
  setting up this lemma, we will let $f$ be the zero function and
  $p_{\mathbb{F}}$ be the zero polynomial.

  From the lemma, we know that there is some $B \in \{0, 1\}^{n}$ with
  $\abs{B} < 2^{n}$ such that for all $f'$ agreeing with $f$ there exists a
  series of $p_{\mathbb{F}}'$ extending $f'$ and agreeing with $p_{\mathbb{F}}$
  on $\mathcal{Y}_{\mathbb{F}}$. As such, if we pick any $w \in \{0, 1\}^{n} \setminus B$, then the
  function $f'(x) = [x = w]$ agrees with $f$ on $B$, and thus we know that there
  exists a series of $p_{\mathbb{F}}'$ that agree with the zero polynomial on
  $\mathcal{Y}_{\mathbb{F}}$ and each non-$w$ Boolean point.

  Now, we know that such a solution exists, and \cref{eqn:p-prime} gives us an
  explicit formula for our $A_{n_{i},\mathbb{F}}$; thus, we know that this is in
  fact computable. Since this algorithm is simply for \emph{constructing} the
  language, we do not care about time or space complexity, so the fact that it
  is computable is enough. In terms of finding the $w$ we need, we can simply
  iterate try the construction for each $w \in \{0, 1\}^{n}$ and stop as soon as
  we are able to construct each polynomial.

  The other component of soundness is determining how we can run $P_{i}$ with
  the extension oracle $\tilde{A}$ when $\tilde{A}$ is not yet fully
  constructed. What we do is when simulating $P_{i}$, we assume that any
  $\tilde{A}_{n_{i},\mathbb{F}}$ that we have not yet queried returns zero on
  all queried inputs. We then make sure that any time we set an
  $\tilde{A}_{n_{i},\mathbb{F}}$, it also returns zero on any point that we
  queried. Further, we ensure that each $n_{i}$ is large enough that no previous
  machine would have queried any string of length $n_{i}$ on its respective
  input; ergo modifying these polynomials would not have any affect on their
  output.

  Next, we show that $L(A)$ is not in $\P^{\tilde{A}}$. As we did in
  \cref{thm:p-np-nrel}, the idea is that for each polynomial-time machine
  $P_{i}$, that machine will return the incorrect result on the string
  $0^{n_{i}}$. We do this in \cref{alg:construct-a-tilde} by simulating $P_{i}$
  on the input, and then adjusting $\tilde{A}$ based on its output. We separate
  this into two cases: the case where $P_{i}^{\tilde{A}}$ rejects $0^{n_{i}}$,
  and the case where it accepts. We shall begin with the case where it accepts.

  When $P_{i}^{\tilde{A}}$ accepts, we want to ensure that no strings of length
  $n_{i}$ is in $A$. The unique low-degree extension of the zero function is the
  zero polynomial; hence, we set $\tilde{A}_{n_{i},\mathbb{F}}$ to be $0$ for
  all $\mathbb{F}$. This ensures $\tilde{A}(x) = 0$ for all
  $x \in \{0, 1\}^{n_{i}}$, and thus $A \cap \{0, 1\}^{n_{i}} = \varnothing$. This
  means $0^{n_{i}} \notin L(A)$ and thus $P_{i}^{\tilde{A}}$ is incorrect.

  When $P_{i}^{\tilde{A}}$ accepts, we want to make sure that there is at least
  some string $w \in A \cap \{0, 1\}^{n_{i}}$, but also to make sure that any
  polynomials we add have their values align with what $P_{i}^{\tilde{A}}$
  already saw. As we mentioned earlier, we know that such a polynomial exists,
  and thus we construct it. Since our constructed polynomials tell us that
  $w \in A$, it follows that $0^{n_{i}} \in L(A)$ and hence $P_{i}^{\tilde{A}}$ is
  incorrect there as well.

  Since our argument earlier told us that none of the $P_{i}$ machines would
  have their output affected by any of the polynomials modified outside of the
  corresponding iteration $i$, it follows that no machine $P_{i}$ could
  recognize $L(A)$. Since $P_{i}$ includes every machine recognizing a
  $\P^{\tilde{A}}$ language, it follows that $L(A) \notin \P^{\tilde{A}}$.
\end{proof}

\section{Arithmetization algebrizes}\label{sec:arith-algebrizes}

% TODO

\chapter{Interactive proof systems}\label{chap:ips}

Interactive proof systems are models of computation that involve multiple Turing
machines exchanging messages between each other. In general, the machines are
split into two categories: those that are computationally unbounded but
untrustworthy (the provers), and those that are bounded but trustworthy (the
verifiers). The ``goal'' of the system is to convince the verifier of whether or
not the string is in the language. These systems almost always use randomness as
part of their design: for this reason, almost all of the bounds are ``with high
probability'' bounds and not complete mathematical certainty.

Interactive proof systems turn out to be surprisingly powerful---while the
verifier only runs in polynomial time, it turns out that the interaction with
the untrustworthy computer is still enough to boost the power significantly.

\section{Single-prover systems}\label{sec:zero-prover}

The simplest form of an interactive proof system is when there is one prover and
one verifier.

\begin{defn}[{\cite[Def.\ 4.2.1]{Go01}}]\label{def:interactive-tm}\index{Turing machine!interactive}
  An \emph{interactive Turing machine} is a deterministic multi-tape Turing
  machine with the following tapes:
  \begin{itemize}
    \item Input tape (read-only)
    \item Output tape (write-only)
    \item Two communication tapes (one read-only, one write-only)
    \item One-cell switch tape (read-write)
    \item Work tape (read-write)
  \end{itemize}
  In addition to these tapes, an interactive TM has a single bit $\sigma \in \{0, 1\}$
  associated with it, called its \emph{identity}. When the content of the switch
  tape is not equal to the machine's identity, it performs no computation and is
  called \emph{idle}.
  % TODO: Finish
\end{defn}

Of course, a single interactive Turing machine is not worth much: in order to do
work with these we need to define how a pair of them interact. The chief
mechanism of interacting Turing machines is that of shared tapes. Shared tapes
are tapes where any modifications can be seen by both Turing machines
immediately. While the tapes themselves are shared, the \emph{heads} are not:
the two machines are perfectly capable of looking at different entries at the
same time.

\begin{defn}[{\cite[Def.\ 4.2.2]{Go01}}]\label{def:linked-tms}\index{Turing machine!linked pair}
  A pair of interactive Turing machines $(M, N)$ are \emph{linked} if the
  following are true:
  \begin{enumerate}
    \item The identity of $M$ is distinct from the identity of $N$.
    \item The switch tapes of $M$ and $N$ coincide (i.e., writing to one affects
          the value in both).
    \item The read-only communication tape of $M$ coincides with the write-only
          communication tape of $N$.
    \item The read-only communication tape of $N$ coincides with the write-only
          communication tape of $M$.
  \end{enumerate}
\end{defn}

\begin{figure}[htbp]
  \centering
  \begin{tikzpicture}[>={Stealth[scale=1.25]}]
    % TODO: Make pretty (esp. colors)
    \draw (0,0) rectangle ++(3,3) node [pos=0.5] {$\id = 0$};
    \draw (8,0) rectangle ++(3,3) node [pos=0.5] {$\id = 1$};
    \draw (5.25,4) rectangle ++(0.5,0.5) node [pos=0.5] {$\sigma$};
    \draw[<->] (5.25,4) -- (3,3);
    \draw[<->] (5.75,4) -- (8,3);
    \foreach \y in {0,...,4} {
      \pgfmathsetmacro{\yvar}{2.75 - 0.5 * \y}
      \draw (4.5,\yvar) rectangle ++(0.5,-0.5);
      \draw (6,\yvar) rectangle ++(0.5,-0.5);
    }
    \path (4.5,-0.25) rectangle ++(0.5,0.5) node [pos=0.5] {$\vdots$};
    \path (6,-0.25) rectangle ++(0.5,0.5) node [pos=0.5] {$\vdots$};
    \draw[->] (3,1.5) -- (4.5,1.5);
    \draw[->] (8,1.5) -- (6.5,1.5);
    \draw[<-] (3,2.5) to [bend left] (6,2.75);
    \draw[<-] (8,2.5) to [bend right] (5,2.75);
    \foreach \x in {0,...,9} {
      \pgfmathsetmacro{\xvar}{3 + 0.5 * \x}
      \draw (\xvar,5) rectangle ++(0.5,0.5);
    }
    \draw[->] (3,5) to (1.5,3);
    \draw[->] (8,5) to (9.5,3);
    \foreach \x in {0,...,9} {
      \pgfmathsetmacro{\xvar}{3 + 0.5 * \x}
      \draw (\xvar,-2.5) rectangle ++(0.5,0.5);
    }
    \draw[->] (1.5,0) to (3,-2);
    \draw[->] (9.5,0) to (8,-2);
    \foreach \y in {0,...,4} {
      \pgfmathsetmacro{\yvar}{2.75 - 0.5 * \y}
      \draw (-2,\yvar) rectangle ++(0.5,-0.5);
      \draw (12.5,\yvar) rectangle ++(0.5,-0.5);
    }
    \path (-2,-0.25) rectangle ++(0.5,0.5) node [pos=0.5] {$\vdots$};
    \path (12.5,-0.25) rectangle ++(0.5,0.5) node [pos=0.5] {$\vdots$};
    \draw[<->] (0,1.5) -- (-1.5,1.5);
    \draw[<->] (11,1.5) -- (12.5,1.5);
  \end{tikzpicture}
  \caption{A linked pair of interactive Turing machines}\label{fig:linked-pair}
\end{figure}

We include a diagram of how a linked pair of Turing machines interact and share
tape as \cref{fig:linked-pair}. The arrows point in the direction data is able
to flow: read-only tapes have an arrow pointing from them and write-only tapes
have an arrow pointing to them.

\begin{defn}[{\cite[Def.\ 4.2.2]{Go01}}]\label{def:joint-comp}\index{joint computation}
  The \emph{joint computation} of a linked pair of interactive Turing machines
  $(M, N)$ is, on a common input string $x$, is the series of computation states
  for both $M$ and $N$ when each is given $x$ as its initial input tape and when
  the initial value of the shared switch tape is $0$. The joint computation
  halts when either machine halts and the halting machine is not idle.
\end{defn}

% TODO: Diagram here of "flow" of joint computation

We will denote the joint computation of machines $M$ and $N$ on input $x$ by
$\ang{M, N}(x)$. % TODO: Talk about how this is a random variable

Now that we have a model for letting two machines talk to each other, we can
define the requirements for an interactive proof. To do this, we need three
pieces: to restrict the complexity of the verifier (lest it simply compute the
problem itself), to require the verifier to generally accept whenever the input
is in the language, and to require the verifier to generally reject whenever the
input is not in the language.

\begin{defn}[{\cite[Def.\ 4.2.4]{Go01}}]\label{def:ips}\index{interactive proof!single-prover}
  An \emph{interactive proof system} is a pair of interactive machines $(P, V)$
  such that $V$ is polynomial-time and the following holds:
  \begin{itemize}
    \item \emph{Completeness}: For every $x \in L$,
          \[
            \mathbb{P}[\ang{P, V}(x) = 1] \ge \frac{2}{3}.
          \]
    \item \emph{Soundness}: For every $x \notin L$ and every interactive machine $B$,
          \[
            \mathbb{P}[\ang{B, V}(x) = 1] \le \frac{1}{3}.
          \]
  \end{itemize}
\end{defn}

While we require our system to be correct at least $2/3$ of the time, our choice
of probability is actually somewhat arbitrary, so long as it is at least $50\%$.
This is because with a greater than $50\%$ chance of success, we can simply run
the checker multiple times and take the majority vote, which will allow us to
get the probability arbitrarily high. Since this iteration is for a fixed number
of times, it will only linearly scale the runtime and thus it does not affect
whether our algorithm is an interactive proof system.

An additional thing to note is that for the soundness clause, we require the
inequality to hold for \emph{any} interactive machine $B$, and not just our
chosen machine $P$. This is important---it says that our verifier cannot be
``fooled'' reliably by a lying machine, so long as $x$ is not in the language
$L$. In practice, what this means is that if the verifier has reason to believe
that the machine it is interacting with is not $P$, then it should always reject
immediately, as we do not care what happens with an arbitrary machine when
$x \in L$. A consequence of this is that if $V$ ever recieves back
improperly-formatted or nonsense input from its prover, it will reject
immediately. Similarly to what we do for ordinary Turing machines parsing their
input, we will not explicitly write out that $V$ should reject if it recieves
a poorly-formatted response, as it serves little but to provide clutter.

As with all of our interactive-proof variants, we will also define a complexity
class corresponding to the set of languages with the given proof. Once we have a
complexity class, we will be able to work with it in the same way we have been
all the ``standard'' classes like $\P$ or $\NSPACE$.

\begin{defn}[{\cite[Def.\ 4.2.5]{Go01}}]\label{def:ip}\index{IP@$\IP$}
  The class $\IP$ is the class of all languages that have an interactive proof
  system.
\end{defn}

Now that we have seen the formal definition of an interactive proof, let us
illustrate the formality with an example. To do so, consider the following
theorem:

\begin{thm}[{\cite{Pra75}}]\label{thm:primes-in-ip}\index{Prime@$\Prime$}
  The language $\Prime = \{p \mid p \text{ is prime}\}$ is in $\IP$.
\end{thm}
% NOTE: See also A Formalization of Pratt’s Primality Certificates (Wimmer, Noschinski)

\begin{algorithm}[H]
  % TODO
  \caption{An interactive proof for the language $\Prime$}\label{alg:prime-ip}
\end{algorithm}

Once we have a complexity class, the question arises of how it relates to other
complexity classes. For $\IP$, Adi Shamir proved in 1992~\cite{Sha92} that a
language has a standard interactive protocol if and only if it is in $\PSPACE$.

% TODO: How much do I want to talk about this? (Scope creep)
\begin{thm}[{\cite{Sha92}}]\label{thm:ip-is-pspace}
  $\IP = \PSPACE$.
\end{thm}

\begin{proof}
  % TODO
\end{proof}

\section{Multi-prover systems}\label{sec:multi-prover}

We have now seen quite a bit of single-prover interactive proofs. Having seen
this, the question might arise of how things would change if we were to add more
machines. Since our verifiers are trusted, increasing the number of verifiers is
not useful since any pair of verifiers we would simply be able to simulate with
a single verifier working twice as hard (which would keep it polynomial).
However, increasing the number of provers to two turns out to give us more power
than we would get with a single prover.

% TODO: Should I define this for any n instead of just 2?
\begin{defn}[{\cite[Def.\ 4.11.2]{Go01}}]\label{def:mps}\index{interactive proof!multi-prover}
  A \emph{multi-prover interactive proof system} is a triplet of interactive
  machines $(P_{1}, P_{2}, V)$ such that $P_{1}$ and $P_{2}$ cannot communicate,
  $V$ is probabilistic polynomial-time, and
  \begin{itemize}
    \item \emph{Completeness}: For every $x \in L$,
          \[
            \mathbb{P}[\ang{P_{1}, P_{2}, V}(x) = 1] \ge \frac{2}{3}.
          \]
    \item \emph{Soundness}: For every $x \notin L$ and every pair of interactive
          machines $B_{1}$ and $B_{2}$,
          \[
            \mathbb{P}[\ang{B_{1}, B_{2}, V}(x) = 1] \le \frac{1}{3}.
          \]
  \end{itemize}
\end{defn}

The above definition should look rather similar to \cref{def:ips}; the only
difference is now we have two provers instead of just one. The fact that the two
provers cannot communicate is important: if they could, they would be able to
``strategize''; that is, agree on a joint plan to make sure that their responses
agree with each other. Since our provers are not required to be computationally
bounded, if they could communicate it would be no different than simply having
one prover. However, since the two provers cannot communicate, we gain some
information from times where they lie in \emph{different} ways: where each
prover individually could say something plausible, but in combination, the
provers' responses contradict.

\begin{defn}\label{def:mip}\index{MIP@$\MIP$}
  The class $\MIP$ is the class of languages that have a multi-prover
  interactive proof system.
\end{defn}

\begin{lemma}\label{lem:ip-in-mip}
  $\IP \subseteq \MIP$.
\end{lemma}

% https://www.math.toronto.edu/swastik/courses/rutgers/topics-S17/lec9.pdf
\begin{thm}[{\cite{BFL90}}]\label{thm:mip-is-nexp}
  $\MIP = \NEXP$.
\end{thm}

\begin{proof}
  % TODO
  We showed in \cref{thm:o3sat-nexp-complete} that $\OSAT$ is $\NEXP$-complete,
  so all we need is to demonstrate that $\OSAT$ has a multi-prover system.
\end{proof}

Having seen how much more powerful systems become with two provers, one might
wonder what would happen if we were to add a third. Unfortunately, it turns out
that a third prover is no more powerful than just having two. We formalize this
below; because we do not get any benefit from three provers we will not work
with three-prover systems at all in this paper beyond this proof.

% FIXME: Cite (where did this come from?)
\begin{thm}\label{thm:mip-unchanged}
  If we redefine $\MIP$ to have $m(n) = \poly(n)$ provers on an input of size
  $n$, the class is unchanged.
\end{thm}

\begin{proof}
  % TODO
\end{proof}

\section{Zero-knowledge proofs}\label{sec:zero-knowledge}

Zero-knowledge proofs are a variant of interactive proofs that have certain
cryptographic requirements. What we care about is the idea that zero-knowledge
proofs transmit \emph{no knowledge} other than precisely the statement trying to
be proved. As an example, if the statement that you are trying to prove is ``I
have an instance of $X$'', the conceptually-easiest way to prove it would be to
produce the aforementioned instance. However, this would not be zero-knowledge
since it also transmits the knowledge of exactly what your instance of $X$ is.

The way we mathematically define zero-knowledge is a little tricky. The way we
demonstrate that the proof is zero-knowledge is by creating a simulator $S_{V}$
for each possible verifier $V$: a machine in $\P$ that \emph{by itself} can
reproduce the entire message log between $P$ and $V$ for any input.

This definition shows that no knowledge has been released because we are able to
reproduce all the public information of the proof with relatively little work.
Having said that, it is not particularly obvious that there are \emph{any}
languages that are outside of $\P$ with perfect zero-knowledge
proofs.\footnote{To some extent, showing that there are languages \emph{truly}
  outside of $\P$ would require a proof that $\P \ne \NP$ (which is unfortunately
  beyond the scope of this paper), but there are lots of languages strongly
  believed to be outside of $\P$ with zero-knowledge proofs.} Despite this, it
turns out that these languages do in fact exist (and are reasonably common).
Abstractly, the idea behind why many of these work is that the verifier can
perform a transformation on some random value, such that undoing the
transformation and reliably recovering the original value is only possible with
knowledge of the language. However, a simulator would have access to the
randomly-chosen value, and thus it could construct a response immediately with
no reference to the problem to be solved.

\begin{defn}[{\cite[Def.\ 4.3.1]{Go01}}]\label{def:zero-knowledge}\index{perfect zero-knowledge}\index{perfect simulator}
  A proof system $(P, V)$ for a language $L$ is \emph{perfect zero-knowledge} if
  for each probabilistic polynomial-time interactive machine $V^{*}$ there
  exists a probabilistic polynomial-time ordinary machine $M^{*}$ such that for
  every $x \in L$ we have the following conditions hold:
  \begin{enumerate}
    \item With probability at most $1/2$ , on input $x$, machine $M^{*}$ outputs
          a special symbol denoted $\bot$ (i.e.\ $\mathbb{P}[M^{*}(x) = \bot] \le 1/2$).
    \item Let $m^{*}(x)$ be the random variable such that
          \begin{equation}
            \mathbb{P}[m^{*}(x) = \alpha] = \mathbb{P}[M^{*}(x) = \alpha \mid M^{*}(x) \ne \bot]
          \end{equation}
          for all $\alpha$. That is, let $m^{*}(x)$ be the distribution of non-$\bot$
          values of $M^{*}$. Then $\ang{P, V^{*}}(x)$ and $m^{*}(x)$ are
          identically distributed for all $x \in L$.
  \end{enumerate}
  In this case, we say the machine $M^{*}$ is a \emph{perfect simulator} for the
  interaction of $V^{*}$ with $P$.
\end{defn}

As with other interactive proof systems, zero-knowledge proofs are
probabilistic; in particular this means they do \emph{not} function as proofs in
the mathematical sense.

\begin{defn}[{\cite[Def.\ 4.3.5]{Go01}}]\label{def:pzk}\index{PZK@$\PZK$}
  The class $\PZK$ is the class of all languages with a perfect zero-knowledge
  proof system.
\end{defn}

% TODO: Example

\section{Probabilistically-checkable proofs}\label{sec:pcp}

So far, all of our computational proofs have focused on the interaction between
two computers, but there exist non-interactive models as well.
Probabilistically-checkable proofs do not use interactive Turing machines, but
instead have access to a ``proof'' that their input is in the given language.
The nontriviality is that the number of bits of the proof we can access is
bounded---simply reading the entire proof will not suffice. For any string in the
language, we ensure there exists a correct proof, which our algorithm must
always recognize accurately. Further, for any string \emph{not} in the language,
the algorithm must reliably (but not necessarily always) reject.

\begin{defn}[{\cite[Def.\ 18.1]{AB09}}]\label{def:prob-check}\index{probabilistically-checkable proof}
  % TODO: Add diagram of this
  Let $L \subseteq \{0, 1\}^{n}$ be a language and $q, r: \mathbb{N} \rightarrow \mathbb{N}$. A
  \emph{$(r(n), q(n))$-verifier} for $L$ is a polynomial-time probabilistic
  algorithm $V$ such that
  \begin{enumerate}
    % TODO: Define adaptive somewhere
    \item When given an input string $x \in \{0, 1\}^{n}$ and random access to a
          string $\pi \in \{0, 1\}^{*}$, $V$ uses at most $r(n)$ random coins and
          makes at most $q(n)$ non-adaptive queries to locations of $\pi$ before
          either accepting or rejecting.
    \item If $x \in L$ then there exists a $\pi \in \{0, 1\}^{*}$ such that $V$ will
          always accept when given input $x$ and random string $\pi$.
    \item If $x \notin L$ then $V$ will reject with probability $\ge 1/2$ for
          \emph{all} random strings $\pi$.
  \end{enumerate}
  We call the random string $\pi$ the \emph{proof}. We denote the output of $V$ on
  input $x$ and proof $\pi$ with $V^{\pi}(x)$.
\end{defn}

So far, all of our proof-complexity classes have just had a single class for all
languages with the proof, regardless of internal complexity, but for
probabilistically-checkable proofs we actually stratify the class further. This
is for multiple reasons: first, we can actually get astonishingly tight bounds
on the parameters for PCPs, and second, because these are ``access'' complexity
(i.e.\ we measure the number of preexisting bits actually read by the
algorithm), they are actually independent of computational model, so the need
for polynomial equivalence is negated.

\begin{defn}[{\cite[Def.\ 18.1]{AB09}}]\label{def:pcp}\index{PCP@$\PCP$}
  For any $q, r: \mathbb{N} \rightarrow \mathbb{N}$, the class $\PCP(q(n), r(n))$ is the class of all
  languages with a $(cq(n), dr(n))$-verifier for some $c, d \in \mathbb{N}$.
\end{defn}

We would be remiss if we were not to mention the PCP theorem, by far the most
important theorem relating to probabilistically-checkable proofs.

\begin{thm}[{$\PCP$ theorem,~\cite{AS98}}]\label{thm:pcp-theorem}\index{PCP theorem@$\PCP$ theorem}
  Any problem in $\NP$ has a probabilistically-checkable proof of constant query
  complexity and using a maximum of $O(\log n)$ random bits, and vice versa.
  Equivalently, $\NP = \PCP(\log n, 1)$.
\end{thm}

\begin{proof}
  % TODO
\end{proof}



\begin{thm}[{\cite{Has97}}]\label{thm:pcp-max-queries}
  Any language in $\NP$ has a $\PCP$ that queries a maximum of $3$ bits of the
  proof and uses $O(\log n)$ random bits.
\end{thm}

\section{Zero-knowledge probabilistically-checkable proofs}\label{sec:pzkpcp}

% TODO: See new (unreviewed) work by Gur, O'Connor, Spooner: A Zero-Knowledge
% PCP Theorem

A zero-knowledge probabilistically-checkable proof is a combination of the ideas
of zero-knowledge proofs (as seen in \cref{sec:zero-knowledge}) and
probabilistically-checkable proofs (as seen in \cref{sec:pcp}). Since we can
model a PCP as an interaction between between a verifier and a proof (instead of
a prover), we can model this interaction as being zero-knowledge as well.

\begin{defn}[{\cite[Def.\ 8.6]{GOS24}}]\label{def:pzkpcp}\index{perfect-zero knowledge PCP}
  A probabilistically-checkable proof system is \emph{zero-knowledge} if there
  exists a probabilistic polynomial-time simulator $S$ such that on input $x$,
  $S$ can simulate every interaction of $V$ with the associated proof of
  $x$.\footnote{To clarify, $S$ does \emph{not} have access to the proof of $x$,
    just $x$ itself.}
\end{defn}

\begin{defn}\label{def:pzkpcp-class}\index{PZK-PCP@$\PZKPCP$}
  The class $\PZKPCP$ is the class of all languages that have a perfect
  zero-knowledge probabilistically-checkable proof.
\end{defn}

% TODO: Example

\section{Interactive probabilistically-checkable proofs}\label{sec:ipcp}

Interactive probabilistically-checkable proofs are a combination of the concepts
of an interactive protocol and a probabilistically-checkable proof. The broad
idea is our proof proceeds in two phases: first, the prover sends a purported
proof to the verifier, after which they engage in an interactive protocol,
during which the verifier can access the proof as an oracle.

\begin{defn}[{\cite[\defaultS 1.1]{KR08}}]\label{def:ipcp}\index{interactive PCP}
  % TODO
  Let $L$ be a language, let $p, q, l: \mathbb{N} \rightarrow \mathbb{N}$, and let $c, s \in [0, 1]$. An
  \emph{interactive probabilistically-checkable proof} for $L$ is an interactive
  protocol as follows:

  \begin{algorithm}[H]
    \KwIn{To both $P$ and $V$: a string $x$ of length $n$}
    \KwIn{To $P$ alone: A string $w$}
    \KwOut{Whether $x \in L$}
    % TODO
    \caption{The IPCP protocol}\label{alg:ipcp-protocol}
  \end{algorithm}
\end{defn}

\begin{defn}\label{def:ipcp-class}\index{IPCP@$\IPCP$}
  The class $\IPCP$ is % TODO
\end{defn}

The tuple-notation we used when talking about the class $\PCP$ (see
\cref{def:pcp}) is rather hard to read when we have this many parameters, and as
such we will us the following clearer notation when talking about the various
bounds on $\IPCP$ algorithms:
\begin{equation*}
  L \in \IPCP\ipcp{r}{\ell}{c}{q}{\varepsilon}
\end{equation*}
to mean the language $L$ is a member of $\IPCP$ with the listed restrictions.

% TODO: Does it make sense to talk about BFL90's IPCP for all of NEXP?

\begin{defn}\label{def:low-deg-ipcp}\index{low-degree IPCP@low-degree $\IPCP$}
  Let $\mathbb{F}$ be a field, and $d, m \in \mathbb{N}$. A \emph{low-degree IPCP} is an
  IPCP instance with the following two properties:
  \begin{enumerate}
    \item The oracle sent by the honest prover $P$ is an $m$-variable
          $\mathbb{F}$-polynomial $Q$ with multidegree no more than $d$ (i.e.\
          $Q \in \mathbb{F}[X_{1, \ldots, m}^{\le d}]$)
    \item Soundness is only required to hold against provers that send oracles
          that are polynomials in $\mathbb{F}[X_{1, \ldots, m}^{\le d}]$.
  \end{enumerate}
\end{defn}

Similarly to what we did with normal $\IPCP$ oracles, we will use the following
notation to talk about low-degree $\IPCP$ instances:
\begin{equation*}
  L \in \IPCP\ldipcp{r}{\ell}{c}{q}{\mathbb{F}[X_{1, \ldots, m}^{\le d}]}{\varepsilon}.
\end{equation*}
We combine all the information about the degree of the oracle into one line
because we have an efficient notation for multidegree-bounded polynomials, and
so that we do not wind up with an exorbitant number of lines in our
notation.\footnote{Having said that, there are still a lot of lines in this
  notation, but this is the best we can do.}

% TODO: Examples

\chapter{Quantum computation}\label{chap:quantum}

\section{Quantum computers}\label{sec:quant-comp}

\begin{defn}\label{def:qubit}\index{qubit}
  A \emph{qubit} is a unit vector in $\mathbb{C}^{2}$.
\end{defn}

\begin{defn}\label{def:operator}\index{operator}
  A \emph{operator} is a linear function $A: V \rightarrow V$ such that
  $v \cdot A \cdot v^{T} \ge 0$ for all $v \in V$.
\end{defn}

\subsection{Measurement}

\begin{defn}\label{def:projective-measure}\index{projective measurement}
  A \emph{projective measurement} is
\end{defn}

\begin{defn}\label{def:povm}\index{positive operator-valued measure}
  A \emph{positive operator-valued measure} is
\end{defn}

\section{Quantum complexity classes}

\begin{defn}\label{def:bqp}\index{BQP@$\BQP$}
  The class $\BQP$ is
\end{defn}

\begin{defn}\label{def:qma}\index{QMA@$\QMA$}
  The class $\QMA$ is
\end{defn}

\section{Quantum interactive proofs}\label{sec:quant-interactive}

\begin{defn}\label{def:mip-star}\index{MIP*@$\MIP*$}
  The class $\MIP*$ is
\end{defn}

% TODO: This is actually worth exploring a *lot* more (it's non-relativizing!)
% See https://scottaaronson.blog/?p=4512
\begin{thm}[{\cite{JNVWY21}}]\label{thm:mip-star-is-re}
  $\MIP* = \RE$.
\end{thm}

\section{Quantum low-multidegree test}

\begin{thm}[{\cite{JNVWY20}}]\label{thm:quantum-low-degree}
  % TODO: What is a projective strategy and (F, d, m)-test?
  There is a universal constant $C > 0$ such that the following holds. Let
  $(\tilde{P}_{1}, \tilde{P}_{2}, \ket{\Psi})$ be a projective strategy that passes
  the $(\mathbb{F}, d, m)$-low-multidegree test with probability at least
  $1 - \varepsilon$. For $i \in \{1, 2\}$, $\alpha \in \mathbb{F}^{m}$, denote by
  $\{A_{i,\alpha}^{z}\}_{z \in \mathbb{F}}$ the measurement applied by $\tilde{P}_{i}$
  upon recieving question $\alpha$. Then there exist projective measurements
  $\{L_{1}^{Q}\}_{Q \in \mathbb{F}[X_{1, \ldots, m}^{\le d}]}$ and
  $\{L_{2}^{Q}\}_{Q \in \mathbb{F}[X_{1, \ldots, m}^{\le d}]}$ such that for
  $v \in \poly(m, d)(d/\abs{\mathbb{F}} + c)^{C}$, the following holds:
  \begin{enumerate}
    \item Consistency with
          $\{A_{1, \alpha}^{z}\}_{z \in \mathbb{F}, \alpha \in \mathbb{F}^{m}}$ and
          $\{A_{1, \alpha}^{z}\}_{z \in \mathbb{F}, \alpha \in \mathbb{F}^{m}}$:
          \begin{align}
            \mathbb{E}_{\alpha \in \mathbb{F}^{m}}\sum_{Q \in \mathbb{F}[X_{1, \ldots, m}^{\le d}]}\sum_{z \in \mathbb{F} \setminus \{Q(\alpha)\}}
            \braopket{\Psi}{A_{1,\alpha}^{z} \otimes L_{2}^{Q}}{\Psi} &\le v, \\
            \mathbb{E}_{\alpha \in \mathbb{F}^{m}}\sum_{Q \in \mathbb{F}[X_{1, \ldots, m}^{\le d}]}\sum_{z \in \mathbb{F} \setminus \{Q(\alpha)\}}
            \braopket{\Psi}{L_{1} \otimes A_{2,\alpha}^{z}}{\Psi} &\le v.
          \end{align}
    \item Consistency of $\{L_{1}^{Q}\}_{Q}$ and $\{L_{2}^{Q}\}_{Q}$:
          \begin{equation}
            \sum_{Q \ne Q' \in \mathbb{F}[X_{1, \ldots, m}^{\le d}]}\braopket{\Psi}{L_{1}^{Q} \otimes L_{2}^{Q}}{\Psi} \le v.
          \end{equation}
  \end{enumerate}
\end{thm}

\begin{proof}
  % TODO
\end{proof}

% TODO: Rename chapter
\chapter{Lifting $\IPCP$ to $\MIP*$}\label{chap:ipcp-to-mip}

\section{Reducing query complexity}

\begin{thm}[{\cite[Prop.\ 9.2]{CFGS22}}]\label{thm:ipcp-one-query}
  There exists a transformation $T$ such that, for every $m, d \in \mathbb{N}$ and finite
  field $\mathbb{F}$, if
  \begin{equation*}
    (P, V) \in \IPCP\ldipcp{r}{\ell}{c}{q}{\mathbb{F}[X_{1, \ldots, m}^{\le d}]}{\varepsilon},
  \end{equation*}
  then $T(P, V)$ recognizes the same language as $(P, V)$ and
  \begin{equation*}
    T(P, V) \in \IPCP\ldipcp{r+1}{\ell}{c+\poly(m,d,q)}{1}{\mathbb{F}[X_{1, \ldots, m}^{\le d}]}{\varepsilon+\frac{mdq}{\abs{\mathbb{F}}-q}}.
  \end{equation*}
\end{thm}

\begin{proof}
  % TODO
\end{proof}

\begin{thm}[{\cite[Prop.\ 9.2]{CFGS22}}]\label{thm:one-query-pzk}
  If $(P, V)$ is perfect zero-knowledge with query bound $b$, then $T(P, V)$
  (where $T$ is the transformation from \cref{thm:ipcp-one-query}) is perfect
  zero-knowledge with query bound $b - (mdq + 1)$.
\end{thm}

\begin{proof}
  % TODO
\end{proof}

\section{A lifting algorithm}

\begin{thm}[{\cite[Lemma 9.1]{CFGS22}}]\label{thm:lift-ipcp-mip}
  Let $L$ be a language, let $m, d, q \in \mathbb{N}$, and let $\mathbb{F}$ be a finite
  field of size $\poly(m, d, q)$ sufficiently large. Then, there exists a
  transformation $T$
  \begin{equation*}
    T: \IPCP\ldipcp{r}{\ell}{c}{q}{\mathbb{F}[X_{1, \ldots, m}^{\le d}]}{\varepsilon}
    \rightarrow \MIP*\mipstar{2}{r+1}{c}{1 - \frac{1}{\poly(m, d)}}.
  \end{equation*}
  such that $(P', V')$ and $T(P', V')$ recognize the same language.

  Further, if the IPCP $(P', V')$ is zero-knowledge with query bound
  $b \ge 2(q+1)md + 3$, then the $\MIP*$ $(P_{1}, P_{2}, V)$ is zero-knowledge.
\end{thm}

\begin{algorithm}[H]
  $V$: Choose a random $r \in \{0, 1\}$\;
  \eIf{$V = 0$}{
    $V$: Perform the low multidegree test from \cref{thm:quantum-low-degree}\;
  }{
    \tcp{IPCP emulation}
    $P_{1}$ and $V$ emulate the interaction of the IPCP $(P'', V'') = T(P', V')$
    (see \cref{thm:ipcp-one-query})\;
    The above emulation generates a uniform $\beta \in \mathbb{F}^{m}$, and a
    $c \in \mathbb{F}$ such that with probability $1 - \varepsilon$, $x \in \mathcal{L}$ if and only if
    $R(\beta) \in c$\;
    $V$: ask $P_{2}$ for an evaluation of $R$ at $\beta$\;
    $P_{2}$: reply with an element $z \in \mathbb{F}$\;
    $V$: accept if and only if $c = z$\;
  }
  \caption{Construction of a $\MIP*$ from an IPCP~\cite[Construction 2]{CFGS22}}\label{alg:mip-from-ipcp}
\end{algorithm}

\subsection{Soundness of \cref{alg:mip-from-ipcp}}

\subsection{\Cref{alg:mip-from-ipcp} preserves zero-knowledge}

\begin{algorithm}[H]
  % TODO
  \caption{A simulator for \cref{alg:mip-from-ipcp}~\cite[\defaultS 9.4]{CFGS22}}\label{alg:sim-mip}
\end{algorithm}

% TODO: Rename chapter
\chapter{Low-degree $\IPCP$ with zero-knowledge}\label{chap:ipcp-zero-knowledge}

\section{AQC of polynomial summation}\label{sec:aqc-poly-sum}

\begin{lemma}[{\cite[Lemma 12.1]{CFGS22}}]
  Let $\mathbb{F}$ be a field, $m, k, d, d' \in \mathbb{N}$, and $G, K, L$ be finite
  subsets of $\mathbb{F}$ such that $K \subseteq L$, $d' \ge \abs{G} - 2$, and
  $\abs{K} = d + 1$. If $S \subseteq \mathbb{F}^{m+k}$ is such that there exist matrices
  $C \in \mathbb{F}^{L^{m} \times \ell}$ and $D \in \mathbb{F}^{S \times \ell}$ such that for
  all $Z \in \mathbb{F}[X_{1, \ldots, m}^{\le d}, Y_{1, \ldots, k}^{\le d'}]$ and all
  $i \in \{1, \ldots, \ell\}$
  \begin{equation}
    \sum_{\alpha \in L^{m}}C_{\alpha,i}\sum_{y \in G^{k}}Z(\alpha, y) = \sum_{q \in S}D_{q,i}Z(q),
  \end{equation}
  then $\abs{S} \ge \rk(BC)(\min(d' - \abs{G} + 2, \abs{G}))^{k}$, where
  $B \in \mathbb{F}^{K^{m} \times L^{m}}$ is such that the column of $B$ indexed by $\alpha$
  represents $Z(\alpha)$ in the basis $\{Z(\beta) \mid \beta \in K^{m}\}$.
\end{lemma}

\begin{proof}
  % TODO
\end{proof}

\begin{cor}[{\cite[Corollary 12.2]{CFGS22}}]
  Let $\mathbb{F}$ be a finite field, $G \subseteq \mathbb{F}$, and $d, d' \in \mathbb{N}$ with
  $d' \ge 2(\abs{G} - 1)$. If $S \subseteq \mathbb{F}^{m+k}$ is such that there exist
  $(c_{\alpha})_{\alpha \in \mathbb{F}^{m}}$ and $(d_{\beta})_{\beta \in \mathbb{F}^{m+k}}$ such that
  \begin{enumerate}
    \item for all $Z \in \mathbb{F}[X_{1, \ldots, m}^{\le d}, Y_{1, \ldots, k}^{\le d}]$ it
          holds that
          \begin{equation}
            \sum_{\alpha \in \mathbb{F}^{m}}c_{\alpha}\sum_{y \in G^{k}}Z(\alpha, y) = \sum_{q \in S}d_{q}Z(q),
          \end{equation}
          and
    \item there exists $Z' \in \mathbb{F}[X_{1, \ldots, m}^{\le d}, Y_{1, \ldots, k}^{\le d}]$
          such that
          \begin{equation}
            \sum_{\alpha \in \mathbb{F}^{m}}c_{\alpha}\sum_{y \in G^{k}}Z'(\alpha, y) = 0,
          \end{equation}
  \end{enumerate}
  then $\abs{S} \ge \abs{G}^{k}$.
\end{cor}

\begin{proof}
  We will leverage \cref{thm:lin-indep-stat-indep} that we proved earlier. Now,
  consider the vector space
  \begin{equation} % FIXME: Holy cow this is *ugly* (and exceptionally unclear)
    \mleft\{
      \mleft(
        (Z(\gamma))_{\gamma \in \mathbb{F}^{m+k}}, \mleft(\sum_{y \in G^{k}}Z(\alpha, y)\mright)_{\alpha \in \mathbb{F}^{m}}
      \mright)
      \middlemid
      Z \in \mathbb{F}[X_{1, \ldots, m}^{\le d}, Y_{1, \ldots, k}^{\le d'}]
    \mright\}
  \end{equation}
  This is a vector space over $\mathbb{F}$ with domain
  $\mathbb{F}^{m+k} \cup \mathbb{F}^{m}$. % FIXME: What on earth do they mean "domain"?
  % TODO
\end{proof}

\begin{cor}[{\cite[Corollary 12.3]{CFGS22}}]
  Let $\mathbb{F}$ be a finite field, $G \subseteq \mathbb{F}$, and $d, d' \in \mathbb{N}$ with
  $d' \ge 2(\abs{G} - 1)$. Let $Q$ be a subset of $\mathbb{F}^{m+k}$ with
  $\abs{Q} \le \abs{G}^{k}$, and let $Z$ be uniformly random in
  $\mathbb{F}[X_{1, \ldots, m}^{\le d}, Y_{1, \ldots, k}^{\le d'}]$. Then, the random
  % FIXME: Better notation for these
  variables $(\sum_{y \in G^{k}}Z(\alpha, y))_{\alpha \in \mathbb{F}^{m}}$ and $(Z(q))_{q \in Q}$
  are independent.
\end{cor}

\begin{proof}
  % TODO
\end{proof}

% TODO: Additional theorem that links this to algebraic query complexity somehow

\section{The sumcheck problem}

\begin{defn}[{\cite{LFKN92}}]\index{sumcheck problem}\label{def:sumcheck}
  % TODO: Should this be a theorem instead?
  The \emph{sumcheck problem} is the following problem:
  \begin{quote}
    Let $H$ be a subset of a finite field $\mathbb{F}$, let
    $F \in \mathbb{F}[X_{1, \ldots, m}^{\le d}]$ a polynomial over $\mathbb{F}$, and let
    $a \in \mathbb{F}$. Does $\sum_{x \in H^{m}}F(x) = a?$
  \end{quote}
  For this question, we give $H$ and $a$ to both the prover and verifier, but
  only the prover gets access to $F$ as an algebraic oracle.
\end{defn}

\subsection{A non-zero-knowledge sumcheck protocol}

\begin{algorithm}[H]
  % NOTE: LFKN's algorithm officially calculates if per(A) = s, where per is the
  % permanent of a matrix, but it should be easily modifiable

  % FIXME: Modify to actually be a polynomial & not the permanent
  \KwIn{A polynomial $A$ and number $s$}
  \KwOut{Whether $\per(A) = s$}
  \tcc{Except where specified, this protocol is written from the perspective of
    the verifier $V$.}
  $P$: pick a prime $p$ and convince $V$ that it is prime using
  \cref{alg:prime-ip}\;
  \tcc{All arithmetic in the remainder of this algorithm is done modulo $p$}
  % TODO: Replace with an until loop
  \While{$L \ne \ang{(B, q)}$ for any $1 \times 1$ matrix $B$}{
    \tcp{``Expand'' phase}
    \eIf{$L$ contains exactly one pair $(B, q)$}{
      $r \leftarrow \dim(B)$\;
      \For{$i$ from $1$ to $r$}{
        $B_{i} \leftarrow B_{1,i}$\tcp*[l]{Matrix minor}
        Ask $P$ for the permanent of $B_{i}$\;
        \If{$\sum_{i=1}^{r}b_{1i}q_{i} \ne q$}{
          \KwRet{0}\;
        }
      }
      $L \leftarrow \ang{(B_{1}, q_{1}), \ldots, (B_{r}, q_{r})}$\;
    }{
      \tcp{``Shrink'' phase}
      Pick two pairs $(C, c)$ and $(D, d)$ from $L$\;
      Ask $P$ for the $r+1$ coefficient of $f(x) = \per(C + x(D - C))$\;
      Construct $g(x)$ from that coefficient\; % FIXME: How???
      \If{$g(0) = c$ or $g(1) = d$}{
        \KwRet{0}\;
      }
      Choose random $a \in \mathbb{Z}_{p}$\;
      Send $a$ to $P$\;
      Replace $(C, c)$ and $(D, d)$ with $(C + a(D - C), g(a))$\;
    }
  }
  \KwRet{$q = \per(B)$}\;
  \caption{The standard sumcheck protocol~\cite[Thm.\ 1]{LFKN92}}\label{alg:sumcheck-std}
\end{algorithm}

\subsection{A weakly zero-knowledge sumcheck protocol}

\begin{algorithm}[H]
  \caption{A weakly zero-knowledge sumcheck protocol~\cite{BCFGRS17}}\label{alg:sumcheck-wzk}
\end{algorithm}

\subsection{Making the sumcheck protocol zero-knowledge}

% \begin{thm}[{\cite{BCFGRS17}}]
%   There exists a probabilistic algorithm $\mathcal{A}$ such that, for every finite field
%   $\mathbb{F}$, $m, d \in \mathbb{N}$, $H \subseteq \mathbb{F}$, subset
%   $S = \{(\alpha_{1}, \beta_{1}), \ldots, (\alpha_{\ell}, \beta_{\ell})\} \subseteq \mathbb{F}^{\le m} \times \mathbb{F}$,
%   and $(\alpha, \beta) \in \mathbb{F}^{\le m} \times \mathbb{F}$,
%   \begin{equation}
%     \mathbb{P}[\mathcal{A}(\mathbb{F}, m, d, H, S, \alpha) = \beta] =
%     \underset{R \leftarrow \mathbb{F}[X_{1, \ldots, m}^{\le d}]}{\mathbb{P}}\mleft[R(\alpha) = \beta \middlemid
%       \begin{matrix}
%         R(\alpha_{1}) = \beta_{1} \\
%         \vdots \\
%         R(\alpha_{\ell}) = \beta_{\ell}
%       \end{matrix}
%     \mright]
%   \end{equation}
%   Further, $\mathcal{A}$ runs in time
%   $m(d\ell\abs{H} + d^{3}\ell^{3})\poly(\log(\abs{\mathbb{F}})) = \ell^{3}\poly(m, d, \abs{H}, \log(\abs{\mathbb{F}}))$.
% \end{thm}

% \begin{proof}
%   % TODO
% \end{proof}

% TODO: Cite location (\S 13 somewhere)
\begin{thm}[{\cite{CFGS22}}]\index{zero-knowledge sumcheck protocol}\label{thm:zk-sumcheck}
  % TODO
  There exists a zero-knowledge variant of \cref{alg:sumcheck-wzk}.
\end{thm}

\begin{proof}
  % TODO
\end{proof}

\begin{algorithm}[H]
  % TODO: Where exactly does G come from and who knows about its contents?
  \KwIn{An instance $(H, a)$ to both $P$ and $V$}
  \KwIn{A polynomial $F \in \mathbb{F}[X_{1, \ldots, m}^{\le d}]$ as an oracle to $P$}
  \KwOut{Whether $\sum_{x \in H^{m}}F(x) = a$}
  $P$: draw uniformly random polynomials
  $Z \in \mathbb{F}[X_{1, \ldots, m}^{\le d}, Y_{1, \ldots, m}^{\le 2\lambda}]$ and
  $A \in \mathbb{F}[Y_{1, \ldots, k}^{\le 2\lambda}]$\;
  $P$: send the polynomial
  \[
    O(W, X, Y) = W \cdot Z(X, Y) + (1 - W) \cdot A(Y)
  \]
  to $V$\;
  \tcp{Note that $Z = O(1, \cdot)$ and $A = O(0, 0, \cdot)$, so $V$ can use both $Z$ and
    $A$ later}
  $P$: send $z = \sum_{a \in H^{m}}\sum_{\beta \in G^{k}}Z(\alpha, \beta)$ to $V$\;
  $V$: draw a random element $\rho_{1} \in \mathbb{F} \setminus \{0\}$ and send to $P$\;
  Run the standard sumcheck IP (\cref{alg:sumcheck-std}) on the statement
  $\sum_{\alpha \in H^{m}}Q(a) = \rho_{1}a + z$, where
  \[
    Q(X_{1}, \ldots, X_{m}) = \rho_{1}F(X_{1}, \ldots, X_{m}) + \sum_{\beta \in G^{k}}Z(X_{1}, \ldots, X_{m}, \beta).
  \]
  % TODO: Format this better
  We have $P$ play the prover and $V$ the verifier, with the following
  modification: For $i = 1, \ldots, m$, in the $i$th round, $V$ samples its random
  element $c_{i}$ from the set $I$ instead of from all of $\mathbb{F}$; if $P$
  ever recieves $c_{i} \in \mathbb{F} \setminus I$, it immediately aborts. In particular,
  in the $m$th round, $P$ sends a polynomial
  \[
    g_{m}(X_{m}) = \rho_{1}F(c_{1}, \ldots, c_{m-1}, X_{m}) + \sum_{\beta \in G^{k}}Z(c_{1}, \ldots, c_{m-1}, X_{m}, \beta)
  \]
  for some $c_{1}, \ldots, c_{m-1} \in I$.\;
  $V$: send $c_{m} \in I$ to $P$\;
  $P$: send $w \in \sum_{\beta \in G^{k}}Z(c, \beta)$ to $V$, where $c = (c_{1}, \ldots, c_{m})$\;
  Both: engage in the weak-ZK sumcheck protocol with respect to the claim
  $\sum_{\beta \in G^{k}}Z(c, \beta) = w$, using $A$ as the masking polynomial. If the
  verifier in that protocol rejects, so does $V$\;
  $V$: output the claim $F(c) = \frac{g_{m}(c_{m}) - w}{\rho_{1}}$\;
  \caption{Strong zero-knowledge sumcheck~\cite[Construction 3]{CFGS22}}\label{alg:zk-sumcheck}
\end{algorithm}

\begin{algorithm}[H]
  % TODO
  \caption{An inefficient simulator for
    \cref{alg:zk-sumcheck}~\cite[p.\ 15:33]{CFGS22}}\label{alg:zk-sumcheck-sim}
\end{algorithm}

\begin{algorithm}[H]
  % TODO
  \caption{An efficient variant of
    \cref{alg:zk-sumcheck-sim}~\cite[p.\ 15:34]{CFGS22}}\label{alg:zk-sumcheck-fast}
\end{algorithm}

% TODO: Zero-knowledge sumcheck

\section{Extending the sumcheck algorithm to $\NEXP$}

% FIXME: CFGS22 claims BFL90 has a IPCP for NEXP, but IPCPs weren't invented
% until 18 years after BFL90 was published?

\begin{thm}[{\cite[Thm.\ 14.2]{CFGS22}}]\label{thm:pzkipcp-for-nexp}
  There exists a $c \in \mathbb{N}$ such that for any query-bound function $b(n)$,
  $d(n) \in \Omega(n^{c})$, $m(n) \in O(n^{c}\log(b))$, and any sequence of fields
  $\mathbb{F}(n)$ that are field extensions of $\mathbb{F}_{2}$ with
  $\abs{\mathbb{F}(n)} \in \Omega((n^{c}\log(b))^{4})$,
  \begin{equation*}
    \OSAT \in \IPCP\ldipcp{O(n, b)}{\poly(2^{n}, b)}{\poly(n, \log(b))}{\poly(n, \log(b))}{\mathbb{F}[X_{1, \ldots, m}^{\le d}]}{1/2},
  \end{equation*}
  which is zero-knowledge with query bound $b$.
\end{thm}

\begin{algorithm}[H]
  \Repeat{$\sum_{\beta \in G^{k}}Z(\alpha, \beta) = A(\gamma_{2}(\alpha))$ for all $\alpha \in H^{m_{2}}$}{
    $P$: Draw a random
    $Z \in \mathbb{F}[X_{1, \ldots, m}^{\le \abs{H} + 2}, Y_{1, \ldots, m}^{\le 2\abs{H}}]$\;
  }
  % TODO
  \For{$i \in \{1, 2, 3\}$}{
    $P$ and $V$ implement \cref{alg:sumcheck-wzk} on the claim
    $\sum_{\beta \in H^{k}}Z(c_{i}', \beta) = h_{i}$\;
  }
  \caption{A low-degree IPCP for $\OSAT$~\cite[p.\ 15:36]{CFGS22}}\label{alg:ipcp-o3sat}
\end{algorithm}

\begin{proof}
  % TODO
\end{proof}

\begin{cor}\label{nexp-pzkipcp}
  $\NEXP \subseteq \PZKIPCP$.
\end{cor}

\begin{proof}
  Since $\OSAT$ is $\NEXP$-complete as per \cref{thm:o3sat-nexp-complete}, we
  can perform a polynomial reduction from any other language to $\OSAT$ and then
  run \cref{alg:ipcp-o3sat}.
\end{proof}

% TODO: Rename chapter
\chapter{Zero-knowledge PCPs for $\#\P$}\label{chap:zk-pcp-for-hp}

\begin{thm}[{\cite[Theorem 8.1]{GOS24}}]\label{thm:pzkpcp-for-sat}
  There exists a perfect zero-knowledge probabilistically-checkable proof for
  $\#\SAT$.
\end{thm}

\begin{cor}\label{cor:hashp-subset-pzkpcp}
  $\#\P \subseteq \PZKPCP$.
\end{cor}

\begin{appendices}

\chapter{More on extension polynomials}\label[appendix]{app:ext-poly}

In this appendix, we will work through some of the algebra we mentioned but did
not go into detail about in \cref{sec:polynomial}.

\section{A proof of \cref{eqn:delta-is-delta}}\label{sec:delta-is-delta}

% TODO: Restate preliminaries
Our goal is to demonstrate the following:
\begin{equation}\label{eqn:delta-is-iverson}
  [x = y] = \prod_{i = 1}^{m}\mleft(\sum_{\omega \in H}\mleft(
    \prod_{\gamma \in H \setminus \{\omega\}}\frac{(x_{i} - \gamma)(y_{i} - \gamma)}{(\omega - \gamma)^{2}}
  \mright)\mright)
\end{equation}
for all $x, y \in H^{n}$. We will do this in two cases: one where $x = y$ and one
where $x \ne y$.

First, assume $x \ne y$ (so we want to show $\delta_{y}(x) = 0$). In this case, there
exists at least one $i$ where $x_{i} \ne y_{i}$. For this $i$, for each $\omega$ there
exists some $\gamma \in H \setminus \{\omega\}$ such that either $x_{i} = \gamma$ or
$y_{i} = \gamma$.\footnote{This piece fails in the case where $x_{i} = y_{i}$, since
  if $\omega = x_{i} = y_{i}$ neither of the terms will ever be zero.} As
such, it follows that either $(x_{i} - \gamma) = 0$ or $(\gamma_{i} - \gamma) = 0$. Hence, for
this $i$ the sum will be entirely over zero terms (since there will be at least
one zero term in the product for each $\omega$). As such, this means that the $i$th
term of our outermost product is $0$, and hence the entire product is $0$, as
desired.

When $x = y$ (and so we want to show $\delta_{y}(x) = 1$), the above equation
simplifies to
\begin{equation}
  [x = y] = \prod_{i = 1}^{m}\mleft(\sum_{\omega \in H}\mleft(
    \prod_{\gamma \in H \setminus \{\omega\}}\frac{(x_{i} - \gamma)^{2}}{(\omega - \gamma)^{2}}
  \mright)\mright)
\end{equation}
Whenever $\omega \ne x_{i}$, the innermost product becomes $0$ since there will be a
term where $\gamma = x_{i}$. Hence, we can simplify this further to
\begin{equation}
  [x = y] = \prod_{i=1}^{m}\mleft(\prod_{\gamma \in H \setminus \{\omega\}}\frac{(x_{i} - \gamma)^{2}}{(x_{i} - \gamma)^{2}}\mright).
\end{equation}
Since $\gamma \ne x_{i}$, we can simplify the fraction to $1$; since we have two nested
products it follows that the equation as a whole simplifies to $1$.

\section{Algebra behind \cref{eqn:delta-poly-small}}\label{sec:delta-poly-small}

Our goal is to show that the equation in \cref{eqn:delta-poly} simplifies to
that of \cref{eqn:delta-poly-small} when $H = \{0, 1\}^{n}$.
% TODO

As a refresher, our starting equation has the form
\begin{equation}
  \delta_{y}(x) = \prod_{i = 1}^{m}\mleft(\sum_{\omega \in H}\mleft(
  \prod_{\gamma \in H \setminus \{\omega\}}\frac{(x_{i} - \gamma)(y_{i} - \gamma)}{(\omega - \gamma)^{2}}
  \mright)\mright).
\end{equation}
We start by manually substituting the outer sum:
\begin{equation}
  \delta_{y}(x) = \prod_{i = 1}^{m}\mleft(\mleft(
    \prod_{\gamma \in \{0, 1\} \setminus \{0\}}\frac{(x_{i} - \gamma)(y_{i} - \gamma)}{(\omega - \gamma)^{2}}
    \mright) + \mleft(
    \prod_{\gamma \in \{0, 1\} \setminus \{1\}}\frac{(x_{i} - \gamma)(y_{i} - \gamma)}{(\omega - \gamma)^{2}}
    \mright)
  \mright).
\end{equation}
Next, notice that the inner products are actually each over one term, so we can
manually substitute there:
\begin{equation}
  \delta_{y}(x) = \prod_{i = 1}^{m}\mleft(
    \frac{(x_{i} - 1)(y_{i} - 1)}{(0 - 1)^{2}} +
    \frac{(x_{i} - 0)(y_{i} - 0)}{(1 - 0)^{2}}
  \mright).
\end{equation}
Next, we simplify, taking note that the denominator of both fractions is $1$:
\begin{equation}
  \delta_{y}(x) = \prod_{i = 1}^{m}\mleft((x_{i} - 1)(y_{i} - 1) + x_{i}y_{i}\mright).
\end{equation}
From here, we take advantage of the fact that $y \in \{0, 1\}^{n}$; here we split
our product into two smaller products: one where $y_{i} = 0$ and one where
$y_{i} = 1$.
\begin{equation}
  \delta_{y}(x) = \mleft(\prod_{i:y_{i}=0}(x_{i} - 1)(0 - 1) + 0x_{i}\mright)
  \mleft(\prod_{i:y_{i}=1}(x_{i} - 1)(1 - 1) + x_{i}1\mright).
\end{equation}
Finally, we simplify, bringing us to \cref{eqn:delta-poly-small}.
\begin{equation}
  \delta_{y}(x) = \mleft(\prod_{i:y_{i}=0}(1 - x_{i})\mright)\mleft(\prod_{i:y_{i}=1}x_{i}\mright).
\end{equation}

\chapter{More on \cref{lem:multilinear-is-pspace}}\label[appendix]{app:bug-in-pspace}

\begin{defn}\label{def:alternating-tm}\index{Turing machine!alternating}
  An \emph{alternating Turing machine} is % TODO
\end{defn}

\begin{proof}[{Proof of \cref{lem:multilinear-is-pspace} as written in~\cite{BFL90}}]
  Let $L$ be a $\PSPACE$-robust language. Let $g_{n}(x_{1}, \ldots, x_{n})$ be the
  multilinear extension of the characteristic function of
  $L_{n} = L \cap \{0, 1\}^{n}$. Clearly, $L \in \P^{g}$, where
  $g = \{g_{n} \mid n \ge 0\}$. We will describe an alternating polynomial-time
  Turing machine with access to $L$ computing $g$. First guess the value
  $z = g_{n}(x_{1}, \ldots, x_{n})$. Then existentially guess the linear function
  $h_{1}(y) = g(y, x_{2}, \ldots, x_{n})$ and verify that $h_{1}(x_{1}) = z$. Then
  universally choose $t_{1} \in \{0, 1\}$ and existentially guess the linear
  function $h_{2}(y) = g(t_{1}, y, x_{3}, \ldots, x_{n})$. Keep repeating this
  process until we have specified $t_{1}, \ldots, t_{n}$ and then verify
  $t_{1}, \ldots, t_{n} \in L$. Since a $\PSPACE$ machine can simulate an alternating
  polynomial-time Turing machine, if $L$ is $\PSPACE$-robust then $g$ is
  Turing-reducible to $L$.
\end{proof}

% \begin{thm}[{\cite[Lemma 6.2]{BFL90}}]\label{thm:multilinear-robust}
%   Every $\PSPACE$-robust language has a Turing-equivalent family of multilinear
%   functions over the integers.
% \end{thm}

% \begin{proof}
%   % TODO
%   We use the following algorithm:

%   \begin{algorithm}[H]
%     \KwIn{$x_{1}, \ldots, x_{n} \in \mathbb{F}$} \KwOut{$g_{n}(x_{1}, \ldots, x_{n})$}
%     Existentially guess some
%     $z \in \mathbb{F}$\tcc*{$z = g_{n}(x_{1}, \ldots, x_{n})$}\nllabel{line:def-z}
%     Existentially guess some linear
%     $h_{1}(y)$\tcc*{$h_{1}(y) = g_{n}(y, x_{2}, \ldots, x_{n})$}\nllabel{line:def-h1}
%     Verify $h_{1}(x_{1}) = z$\;
%     \For{$i$ from $2$ to $n$}{
%       Existentially guess some linear
%       $h_{i}(y)$\tcc*{$h_{i}(y) = g_{n}(t_{1}, \ldots, t_{i-1}, y, x_{i+1}, \ldots, y_{n})$}\nllabel{line:def-hn}
%       Verify $h_{i-1}(t_{i-1}) = h_{i}(x_{i})$\;\nllabel{line:verify-hi}
%       Universally choose $t_{i} \in \{0, 1\}$\;
%     }
%     Verify $h_{n}(t_{n}) = [(t_{1}, \ldots, t_{n}) \in L]$\;
%     \Return $z$\;
%     \caption{An algorithm to compute $g_{n}$}\label{alg:compute-gn}
%   \end{algorithm}

%   % TODO: Put info about alternating TMs in thesis?
%   First, we show that this is a polynomial-time alternating Turing machine. Any
%   guess can be made in polynomial time, as can calculating the value of a linear
%   function. We do this a polynomial number of times (specifically linear), so
%   that is in polynomial time. Lastly, calculating whether
%   $(t_{1}, \ldots, t_{n}) \in L$ is in $\PSPACE$, so by a theorem of Chandra, Kozen,
%   and Stockmeyer~\cite[Corollary 3.6]{CKS81}, we can simulate it in polynomial
%   time. Again by that theorem, the fact that \cref{alg:compute-gn} is a
%   polynomial time alternating Turing machine means that the problem it computes
%   is in $\PSPACE$.

%   Next, we need to show that \cref{alg:compute-gn} actually computes the
%   function it purports to. We have put in the comments some equations next to
%   each line in which we existentially guess a value: we shall start by
%   demonstrating that these equations hold for exactly the choices that lead to
%   an accepting configuration.

%   First, we show that the equations in line~\ref{line:def-h1}
%   and~\ref{line:def-hn} hold. First, note that line~\ref{line:def-h1} is just
%   the $i=1$ case of line~\ref{line:def-hn}; hence proving the general case will
%   give us both equations. If we assume
%   \begin{equation}\label{eqn:hi-is-restriction}
%     h_{i}(y) = g_{n}(t_{1}, \ldots, t_{i-1}, y, x_{i+1}, \ldots, y_{n})
%   \end{equation}
%   for all $n$, then the expressions on each side of the equality in the
%   verification on line~\ref{line:verify-hi} will simplify to
%   \begin{equation}
%     g_{n}(t_{1}, \ldots, t_{i-1}, x_{i}, \ldots, x_{n}).
%   \end{equation}
%   Since both sides simplify to the same equation, it follows that they are
%   equal. For the converse, if we assume
%   \begin{equation}
%     h_{i}(y) \ne g_{n}(t_{1}, \ldots, t_{i-1}, y, x_{i+1}, \ldots, y_{n}),
%   \end{equation}
%   for at least one $i$, then either $h_{i-1}(t_{i-1})$ will not be equal to
%   $h_{i}(x_{i})$, in which case line~\ref{line:verify-hi} will fail, or that
%   check will succeed, in which case things get more complex. % TODO

%   Next, we show that the equation in line~\ref{line:def-z} holds. Formatted
%   mathematically, the statement we are making here is that of
%   \cref{lem:multilinear-forces-unique}. Since we have already proved that that
%   statement holds, it follows that line~\ref{line:def-z} does as well. Further,
%   since the equation in line~\ref{line:def-z} is the same as the goal of
%   \cref{alg:compute-gn}, it follows that the algorithm computes the value of
%   $g_{n}$ for arbitrary inputs.
% \end{proof}
\end{appendices}

\printbibliography[heading=bibintoc]{}

\printindex{}

\end{document}
