\documentclass[xcolor={usenames,dvipsnames}]{beamer}

\mode<presentation>
{
  \usetheme{default}      % or try Darmstadt, Madrid, Warsaw, ...
  \usecolortheme{default} % or try albatross, beaver, crane, ...
  \usefonttheme{default}  % or try serif, structurebold, ...
  \setbeamertemplate{navigation symbols}{}
  \setbeamertemplate{caption}[numbered]
}

\hypersetup{
    colorlinks = true,
    urlcolor=blue,
}

\usepackage[english]{babel}
\usepackage{transparent}
\usepackage[style=authoryear]{biblatex}
\usepackage{amssymb}
\usepackage{mathtools}
\usepackage{complexity}
\usepackage{pdfpc}
\usepackage[absolute,overlay]{textpos}

\hypersetup{hidelinks}

\addbibresource{bibliography.bib}

% Tikz stuff.
\usepackage{tikz}
\usepackage{tikz-cd}
\usepackage{pgfplots}
\pgfplotsset{compat=1.16}
%\tikzstyle{ball} = [circle,shading=ball, ball color=black,
%    minimum size=1mm,inner sep=1.3pt]
%\usetikzlibrary{arrows.meta}
%\tikzset{>={Latex[length=2mm,width=1.5mm]}}
\usepgfplotslibrary{fillbetween}
\usetikzlibrary{decorations.pathreplacing}
\tikzset{
      labr/.style={anchor=north, rotate=90, inner sep=.7mm}
    }
\tikzset{
      labl/.style={anchor=south, rotate=90, inner sep=.7mm}
    }
\usetikzlibrary{positioning}
%\usetikzlibrary{graphs,graphs.standard}
\usetikzlibrary{decorations.pathmorphing}
\usetikzlibrary{decorations.pathreplacing}
\usetikzlibrary{arrows.meta,calc}
\usetikzlibrary{bending}
\usetikzlibrary{decorations.markings,shapes.geometric}
\tikzset{->-/.style={decoration={markings, mark=at position #1 with
{\arrow{>}}},postaction={decorate}}}
\tikzset{-|-/.style={decoration={markings, mark=at position #1 with
{\arrow{stealth}}},postaction={decorate}}}
\tikzset{movearrow/.style 2 args ={
    decoration={markings,
      mark= at position {#1} with {\arrow{#2}} ,
    },
    postaction={decorate}
  }
}
\tikzset{<--/.style={decoration={markings, mark=at position #1 with
{\arrow{<}}},postaction={decorate}}}
\tikzstyle{ball} = [circle,shading=ball, ball color=black,
minimum size=1mm,inner sep=1.3pt]
\tikzstyle{bball} = [circle,shading=ball, ball color=blue,
minimum size=1mm,inner sep=1.3pt]
\tikzstyle{miniball} = [circle,shading=ball, ball color=black,
minimum size=1mm,inner sep=0.5pt]
\tikzstyle{redminiball} = [circle,shading=ball, ball color=red,
minimum size=1mm,inner sep=0.5pt]
\tikzstyle{mminiball} = [circle,shading=ball, ball color=black,
minimum size=0.6mm,inner sep=0.1pt]
\tikzstyle{mycircle} = [draw,circle, minimum size=1mm,inner sep=1.3pt]

\DeclareMathOperator{\Span}{\mathrm{Span}}
\DeclareMathOperator{\rk}{\mathrm{rank}}
\DeclareMathOperator{\id}{\mathrm{id}}
\DeclareMathOperator{\tr}{tr}
\DeclareMathOperator{\rspace}{\mathrm{rowspace}}
\DeclareMathOperator{\rrk}{\mathrm{rowrank}}
\DeclareMathOperator{\cspace}{\mathrm{colspace}}
\DeclareMathOperator{\crk}{\mathrm{colrank}}
\DeclareMathOperator{\im}{im}
\DeclareMathOperator{\cok}{cok}
\DeclareMathOperator{\diag}{diag}
\DeclareMathOperator{\GL}{GL}
\DeclareMathOperator{\sgn}{\mathrm{sgn}}
\DeclareMathOperator{\outdeg}{\mathrm{outdeg}}
\DeclareMathOperator{\supp}{\mathrm{supp}}
\DeclareMathOperator{\vol}{\mathrm{vol}}

\definecolor{myblue}{rgb}{0.2,0.2,0.7}
\definecolor{psucolor}{rgb}{0.43,0.55,0.14}
\definecolor{reedcolor}{RGB}{169,14,19}
\setbeamercolor{frametitle}{fg=reedcolor,bg=white}
\setbeamertemplate{itemize item}{\color{reedcolor}$\blacktriangleright$}
\setbeamertemplate{itemize subitem}{\color{reedcolor}$\blacktriangleright$}

\DeclarePairedDelimiter{\abs}{\lvert}{\rvert}

\title[Algebrization]{\bf\huge \textcolor{reedcolor}{Algebrization}}
\author{Patrick Norton}
\date{\textcolor{reedcolor}{December 4, 2024}}
%\titlegraphic{{\includegraphics[height=1.8in]{hyperplane_arrangement.png}}}
\makeatletter
\setbeamertemplate{title page}[default][left,colsep=-4bp,rounded=true,shadow=\beamer@themerounded@shadow]
\makeatother

\begin{document}

% FIXME: Change citation link color

% \usebackgroundtemplate{
%   \vbox to \paperheight{\vfil\hbox to \paperwidth{\hfil
%       {{\transparent{0.1}\includegraphics[height=4.5in]{quintics.png}}}
%   \hfil}\vfil}
% }

\begin{frame}[label=titlepage]
  \titlepage{}
\end{frame}

\usebackgroundtemplate{}

\begin{frame}
  \frametitle{Oracle}
  ``Magic'' function: can solve some problem in a single step

  \bigskip
  Can be any function, but we care when the function is hard to compute
\end{frame}

\begin{frame}
  \frametitle{Relativization}
  % \begin{itemize}
  %   \item A statement about \emph{theorems}: statements of the form $\mathcal{C} = \mathcal{D}$ or
  %         $\mathcal{C} \ne \mathcal{D}$
  %   \item The result $\mathcal{C} = \mathcal{D}$ \emph{relativizes} if $\mathcal{C}^{A} = \mathcal{D}^{A}$ for each
  %         oracle $A$
  %   \item The result $\mathcal{C} \subsetneq \mathcal{D}$ \emph{relativizes} if $\mathcal{C}^{A} \subsetneq \mathcal{D}^{A}$ for each
  %         oracle $A$
  %   \item Example: Since every deterministic TM is also a nondeterministic one,
  %         $\P \subseteq \NP$, and for the same reason $\P^{A} \subseteq \NP^{A}$
  % \end{itemize}

  Statement about \emph{theorems}, not classes or problems

  \bigskip
  \begin{minipage}[t]{0.48\linewidth}
    \begin{center}
      $\mathcal{C} \subseteq \mathcal{D}$ algebrizes:

      \bigskip
      $\mathcal{C}^{A} \subseteq \mathcal{D}^{A}$
    \end{center}
  \end{minipage}\hfill
  \begin{minipage}[t]{0.48\linewidth}
      $\mathcal{C} \nsubseteq \mathcal{D}$ algebrizes:

      \bigskip
      $\mathcal{C}^{A} \nsubseteq \mathcal{D}^{A}$
  \end{minipage}

  \bigskip
  \bigskip
  for all oracles $A$ and extensions $\tilde{A}$
\end{frame}

\begin{frame}
  \frametitle{Relativizing techniques}
  % \begin{itemize}
  %   \item Along with relativizing theorems, \emph{proof techniques} can also
  %         relativize
  %   \item What do we mean by this?
  %   \item A proof technique relativizes if the logic is still valid when you
  %         replace every class $\mathcal{C}$ with $\mathcal{C}^{A}$ for \emph{any} oracle $A$
  %   \item Why do we care?  Well, if you have a proof \emph{technique}
  %         that relativizes, you can only ever use it to prove a \emph{result}
  %         that relativizes
  % \end{itemize}
  Not just theorems, also techniques!

  \bigskip
  Relativizes when logic is still valid

  \bigskip
  Replace $\mathcal{C}$ with $\mathcal{C}^{A}$ everywhere
\end{frame}

\begin{frame}
  \frametitle{Diagonalization}
  Goal: construct a language $L$ not in some class $\mathcal{C}$

  \bigskip
  Idea: go through every machine in $\mathcal{C}$ and make sure it's incorrect for some
  string in $L$

  \bigskip
  Relativizing technique!
\end{frame}

\begin{frame}
  \frametitle{$\P \ne \NP$ does not relativize}
  \begin{itemize}
    \item Idea: find an oracle powerful enough to make the difference between
          $\P$ and $\NP$ disappear
    \item If our oracle $A$ is $\PSPACE$-complete, $\P^{A} = \PSPACE = \NP^{A}$
    % TODO
  \end{itemize}
\end{frame}

\begin{frame}
  \frametitle{$\P = \NP$ does not relativize}
    Time to diagonalize!

    \bigskip
    Goal: construct a language $L$ such that for some oracle $A$,
          $L \in \NP^{A} \setminus \P^{A}$

    \bigskip
    Define
    \[
      L(A) = \{x \in \{0, 1\}^{*} \mid \text{there is } y \in A \text{ such that
      } \abs{y} = \abs{x}\}
    \]
\end{frame}

\begin{frame}
  \frametitle{$L(A) \in \NP^{A}$}
  \begin{textblock*}{0.4\paperwidth}[0,0](0.6\textwidth,10pt)
    \tiny
    $\NP$: languages that can be verified in polynomial time

    $\NP^{A}$: $\NP$ but with access to oracle $A$

    $L(A)$: all strings with the same length as something in $A$
  \end{textblock*}

  Certificate: a string in $A$ of length $n$

  \bigskip
  Verify: Pass certificate to oracle $A$ and verify it outputs 1
\end{frame}

\begin{frame}
  \frametitle{Diagonalizing $A$: definitions}
  \begin{itemize}
    \item Let $\{P_{i}\}$ be a sequence of all oracle machines in $\P$
    \item For each $P_{i}$, let $p_{i}(n)$ be a polynomial upper bound on the
          runtime of $P_{i}$ over all inputs of length $n$
    % \item Define $A(i-1)$ to be all the strings queried by at least one $P_{j}$
    %       on input $0^{j}$ for $j < i$
    \item Next: construct $A$
    \item Do this by constructing a chain
          \[
            A(1) \subseteq \cdots \subseteq A(i) \subseteq \cdots
          \]
    \item Each $A(i)$ corresponds to a machine $P_{i}$
   \end{itemize}
\end{frame}

\begin{frame}
  \frametitle{Diagonalizing $A$}
  \begin{itemize}
    \item For each $P_{i}$:
    \item Let $n_{i} > n_{i-1}$ such that $p_{i}(n_{i}) < 2^{n_{i}}$
    \item This ensures $P_{i}$ will not query some string of length $n_{i}$

    \item Further, it ensures we don't modify anything a previous machine
          queries
    \item Run $P_{i}$ on input $0^{n}$ with oracle $A(i-1)$
    \item Now, we do different things depending on whether $P_{i}$ accepts
  \end{itemize}
\end{frame}

\begin{frame}
  \frametitle{If $P_{i}$ accepts}
  \begin{itemize}
    \item We want to make sure that there are no strings of length $n_{i}$ in
          $A$
    \item Hence, we say $A(i) = A(i-1)$ and thus add nothing new
  \end{itemize}
\end{frame}

\begin{frame}
  \frametitle{If $P_{i}$ rejects}
  \begin{itemize}
    \item This is the tricker half
    \item Since $p_{i}(n_{i}) < 2^{n_{i}}$, we know there is some string $x$ of
          length $n_{i}$ that $P_{i}$ never queried
    \item Hence, if we add that string to $A(i)$, it won't affect the output of
          $P_{i}$
    \item So set $A(i) = A(i - 1) \sqcup \{x\}$
  \end{itemize}
\end{frame}

\begin{frame}
  \frametitle{Algebrization}
  \begin{itemize}
    \item While relativization was enough to rule out all techniques at the
          time, new ones have come about since
    \item Let's rule out all these new ones too
    \item We can think of an oracle as being a collection of functions
          $f_{n}: \{0, 1\}^{n} \rightarrow \{0, 1\}$ for each $n$
    \item What if we extended $f$ to be over a finite field?
  \end{itemize}
\end{frame}

\begin{frame}
  \frametitle{Algebrization}
  \begin{itemize}
    \item Ok, maybe that's a few too many functions
    \item So, let's restrict to polynomials over finite fields
    \item Formal definition: An extension $\tilde{A}$ of an oracle $A$ is a
          collection of polynomials $\tilde{A}_{n,\mathbb{F}}$ such that
          $A_{n}(x) = \tilde{A}_{n,\mathbb{F}}(x)$ for all $x \in \{0, 1\}^{n}$
  \end{itemize}
\end{frame}

\begin{frame}
  \frametitle{Algebrizing results}
  Big difference: only {\color{blue}{one}} oracle gets extended
  \bigskip
  \bigskip

  \begin{minipage}[t]{0.48\linewidth}
    \begin{center}
      $\mathcal{C} \subseteq \mathcal{D}$ algebrizes:

      \bigskip
      $\mathcal{C}^{A} \subseteq \mathcal{D}^{\color{blue}{\tilde{A}}}$
    \end{center}
  \end{minipage}\hfill
  \begin{minipage}[t]{0.48\linewidth}
      $\mathcal{C} \nsubseteq \mathcal{D}$ algebrizes:

      \bigskip
      $\mathcal{C}^{\color{blue}{\tilde{A}}} \nsubseteq \mathcal{D}^{A}$
  \end{minipage}

  \bigskip
  \bigskip
  for all oracles $A$ and extensions $\tilde{A}$
\end{frame}

\begin{frame}
  \frametitle{Algebrizing theorems}
  \begin{itemize}
    \item Also like before, proof techniques can algebrize
    \item A proof technique algebrizes if the logic is still valid when you
          replace every class $\mathcal{C}$ with either $\mathcal{C}^{A}$ or $\mathcal{C}^{\tilde{A}}$, as
          appropriate
  \end{itemize}
\end{frame}

\begin{frame}
  \frametitle{Why algebrization}
  Good news! This is still enough to cause problems
\end{frame}

\begin{frame}
  \frametitle{$\P \ne \NP$ does not algebrize}
  \begin{itemize}
    \item Idea: same as for relativization, we just need to make sure
          $\NP^{\tilde{A}} = \PSPACE = \P^{A}$
    \item From~\cite{BFL90}, if $A$ is $\PSPACE$-complete, then so is
          $\tilde{A}$ so long as each polynomial in $\tilde{A}$ is multilinear

    \item So we know $\NP^{\tilde{A}} = \PSPACE$ and from earlier, we know
          $\P^{A} = \PSPACE$
  \end{itemize}
\end{frame}

\begin{frame}
  \frametitle{$\P = \NP$ does not algebrize}
  \begin{itemize}
    \item Diagonalization returns!
    \item Same as before: construct an $A$ such that $L(A) \in \NP^{A} \setminus \P^{\tilde{A}}$
    \item What's different?
    \item We have to ensure queries ``off the cube'' don't break
  \end{itemize}
\end{frame}

\begin{frame}
  \frametitle{$\P = \NP$ does not algebrize}
  Reuse definitions from before:

  \begin{itemize}
    \item $P_{i}$ a sequence of $P$ machines
    \item Polynomial $p_{i}$ upper bound on running time of $P_{i}$
    \item $n_{i}$ such that $p_{i}(n_{i}) < 2^{n_{i}}$
  \end{itemize}
\end{frame}

\begin{frame}
  \frametitle{$\P = \NP$ does not algebrize}
  \begin{textblock*}{\paperwidth}[0,0](0.6\textwidth,10pt)
    \tiny
    \begin{itemize}
      \item $P_{i}$ a sequence of $P$ machines
      \item Polynomial $p_{i}$ upper bound on running time of $P_{i}$
      \item $n_{i}$ such that $p_{i}(n_{i}) < 2^{n_{i}}$
    \end{itemize}
  \end{textblock*}

  Let $\mathcal{Y}_{\mathbb{F}}$ be the set of points queried by $P_{i}$ for some field
  $\mathbb{F}$

  \bigskip
  The point $w$ is the string we want to be $1$

  \bigskip
  We want:
  \begin{enumerate}
    \item $\tilde{A}_{n,\mathbb{F}}(y) = 0$ for all $y \in \mathcal{Y}_{\mathbb{F}}$
    \item $\tilde{A}_{n,\mathbb{F}}(w) = 1$
    \item $\tilde{A}_{n,\mathbb{F}}(x) = 0$ for all $x \in \{0, 1\}^{n} \setminus \{w\}$
  \end{enumerate}
\end{frame}

\begin{frame}
  \frametitle{References}

  \nocite{AW09}
  \printbibliography{}
\end{frame}

\end{document}
